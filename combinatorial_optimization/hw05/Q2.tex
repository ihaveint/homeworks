\documentclass[a4paper,12pt]{article}
\usepackage{HomeWorkTemplate}
\usepackage{circuitikz}
\usepackage[shortlabels]{enumitem}
\usepackage{hyperref}
\usepackage{tikz}
\usepackage{listings}

\def\Max{\text{بیشینه کن}}
\def\Min{\text{کمینه کن}}
\def\st{\text{\rl{که}}}

\usetikzlibrary{shapes.multipart}

\usepackage{amsmath}
\usepackage{amssymb}
\usepackage{tcolorbox}
\usepackage{xepersian}
\settextfont{XB Niloofar}
\usetikzlibrary{arrows,automata}
\usetikzlibrary{circuits.logic.US}
\usepackage{changepage}
\newcounter{problemcounter}
\newcounter{subproblemcounter}
\setcounter{problemcounter}{1}
\setcounter{subproblemcounter}{1}
\newcommand{\problem}[1]
{
	\subsection*{
		پرسش
%		\arabic{problemcounter} 
%		\stepcounter{problemcounter}
%		\setcounter{subproblemcounter}{1}
		#1
	}
}
\newcommand{\subproblem}{
	\textbf{\harfi{subproblemcounter})}\stepcounter{subproblemcounter}
}


\begin{document}
\handout
{بهینه‌سازی ترکیبیاتی مقدماتی}
{مرتضی علیمی، هانی احمد زاده}
{بهار 1400}
{اطلاعیه}
{سروش زارع}
{97100405}
 {تمرین تحویلی 5}
\problem{2}
ابتدا برنامه‌ی اصلی ($Primal$) را می‌نویسیم:

\begin{align*}
\text{\Min} &\quad  c^T f\\
\text{\st} &\quad  0 \leq f \leq u \\
&\quad \nabla(f)_i = b_i \quad \forall i \in V
\end{align*}

حال برنامه‌ی دوگان ($Dual$) را می‌نویسیم:

\begin{align*}
\text{\Max} &\quad  b^T \pi + u^Ty  \\
\text{\st} &\quad  y_{i,j} + \pi_j - \pi_i \leq c_{i,j} \quad \forall e=ij \in E \\
&\quad y, \pi \geq 0 
\end{align*}

طبق لنگی مکمل، اگر $f^*$ یک جواب شدنی برای $P$ و مجموعه‌ی 
$B^* = \{y^*, \pi^*\}$
یک جواب شدنی برای $D$ باشند، این جواب ها بهینه اند اگر و تنها اگر داشته باشیم:
\begin{align}
	f^*_e > 0 \rightarrow {y^*}_{i,j} + \pi^*_j - \pi^*_i = c_{i,j} \\
	f^*_e < u_e \rightarrow y^*_e = 0
\end{align}
\lemma{تساوی سمت راست (۱) در صورت بهینه بودن $D$ همواره برقرار است و در واقع شرطی اضافه است.}
\proof{}
(فرض خلف) فرض کنید
$B^* = \{y^*, \pi^*\}$
یک جواب بهینه برای $D$ باشد.
اگر $i,j$ موجود باشند به طوری که داشته باشیم
\begin{align*}
{y^*}_{i,j} + \pi^*_j - \pi^*_i < c_{i,j}
\end{align*}
می‌توانیم $y_{i,j}$ را به اندازه‌ی یک $\epsilon > 0$ افزایش دهیم به طوری که همچنان قیود $D$ برقرار باشند و علاوه بر آن، با توجه به مثبت بودن ظرفیت‌ها (درایه‌های $u$)، تابع هدف نیز افزایش پیدا می‌کند که با بهینه بودن $B$ در تناقض است. پس فرض خلف باطل است و لم (۱) اثبات می‌شود.

حال با توجه به لم (۱)، می‌توانیم بنویسیم:
\begin{align*}
	{y^*}_{i,j} = c_{i,j} - \pi^*_j + \pi^*_i
\end{align*}
و در نتیجه شرط (۲) را می‌توانیم به صورت معادل زیر بنویسیم:
\begin{align*}
	\color{blue} f^*_e < u_e \rightarrow c_{i,j} - \pi^*_j + \pi^*_i = 0
\end{align*}
پس توانستیم شروط لنگی مکمل را بدون استفاده از $y^*$ بنویسیم.
\end{document}
