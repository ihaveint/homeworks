\documentclass[a4paper,12pt]{article}
\usepackage{HomeWorkTemplate}
\def\Max{\text{بیشینه کن}}
\def\Min{\text{کمینه کن}}
\def\st{\text{\rl{که}}}

\usepackage{circuitikz}
\usepackage[shortlabels]{enumitem}
\usepackage{hyperref}
\usepackage{tikz}
\usepackage{listings}

\usetikzlibrary{shapes.multipart}

\usepackage{amsmath}
\usepackage{amssymb}
\usepackage{tcolorbox}
\usepackage{xepersian}
\settextfont{XB Niloofar}
\usetikzlibrary{arrows,automata}
\usetikzlibrary{circuits.logic.US}
\usepackage{changepage}
\newcounter{problemcounter}
\newcounter{subproblemcounter}
\setcounter{problemcounter}{1}
\setcounter{subproblemcounter}{1}
\newcommand{\problem}[1]
{
	\subsection*{
		پرسش
%		\arabic{problemcounter} 
%		\stepcounter{problemcounter}
%		\setcounter{subproblemcounter}{1}
		#1
	}
}
\newcommand{\subproblem}{
	\textbf{\harfi{subproblemcounter})}\stepcounter{subproblemcounter}
}


\begin{document}
\handout
{بهینه‌سازی ترکیبیاتی مقدماتی}
{مرتضی علیمی، هانی احمد زاده}
{بهار 1400}
{اطلاعیه}
{سروش زارع}
{97100405}
 {تمرین تحویلی 2}
\problem{2}
\paragraph{\color{red} برای جلوگیری از ابهام، مجموعه‌ی $A$ گفته شده در سوال را با $B$ نشان می‌دهیم. همچنین در این اثبات منظور از ماتریس یک متوالی، ماتریس یک متوالی ستونی نیست و منظور یک متوالی سطری است که اثبات را راحت تر می‌کند.}

\paragraph{الف-}

نقاط مجموعه‌ی $‌B$ را به صورت
$\{p_1, ... p_m\}$
فرض کنید.
کافی است ماتریس $A$ را به این صورت تعریف کنیم که 
$A_{i,j} = 1$
اگر و تنها اگر $p_j \in I_i$ و در غیر اینصورت $A_{i,j} = 0	$ و برنامه ریزی خطی به این صورت می‌شود:
\begin{align*}
\text{\Min} &\quad w^Tx\\
\text{\st} &\quad  Ax \leq k\\
&\quad 0 \leq x_i \leq 1
\end{align*}

که در واقع آرام‌سازی شده‌ی برنامه‌ریزی صحیح زیر است که جواب مساله‌ در حالت صحیح را می‌دهد:
\begin{align*}
\text{\Min} &\quad w^Tx\\
\text{\st} &\quad  Ax \leq k\\
&\quad x_i \in \{0,1\}
\end{align*}
\begin{latin}
	$\square$
\end{latin}
\paragraph{ب-}
فرض می‌کنیم نقاط $p_i$ به ترتیب صعودی مرتب شده‌اند. با این فرض هر سطر ماتریس $A$ که متناظر با یک بازه‌ی $I_i$ است، تعدادی متوالی از نقاط $p_i$ را درون خود دارد و بقیه‌ی نقاط را درون خود ندارد، بنابراین هر سطر $A$ تعدادی متوالی عدد ۱ دارد و بقیه‌ی خانه‌هایش ۰ هستند. 
\definition{ماتریس $A_{n*m}$ را یک متوالی سطری(یا به اختصار یک متوالی ) می‌نامیم، اگر هر سطر آن یک متوالی باشد. یعنی برای هر
$i \in \{1,...,n\}$،
دو عدد مثل
$l_i, r_i \in \{1,...,m\}$
وجود داشته باشند که:
\begin{align*}
	a_{i,j} = \left\{
\begin{array}{ll}
      1 & l_i \leq j \leq r_i \\
      0 & otherwise \\
\end{array} 
\right.
\end{align*}
 }
با توجه به تعریف بالا، ماتریس $A$ ساخته شده، یک متوالی سطری است. ادعا می‌کنیم که این ماتریس تماما تک‌پیمانه‌ای است، که از قضیه‌ی زیر نتیجه‌ می‌شود:
\theorem{اگر $A$ ماتریس یک متوالی سطری باشد، آنگاه $A$ تماما تک‌پیمانه‌ای است.}

\proof{}
ابتدا نیاز‌داریم برخی پیش‌نیاز ها را تعریف کنیم و پس از آن این قضیه به راحتی اثبات می‌شود:
\definition{یک دو رنگ‌آمیزی متساوی از ماتریس $A$، افزار ستون‌های $A$ به دو مجموعه‌ی قرمز و آبی است به طوری که تفاضل جمع‌ستون‌های آبی و جمع ستون‌های قرمز، برداری با مولفه‌های $0$و$-1$و$1$ باشد.
}

\theorem{ماتریس $A$ تماما تک‌‌‌پیمانه‌ای است، اگر و فقط اگر هر زیرمجموعه از ستون‌های $A$ یک دورنگ آمیزی متساوی داشته باشد.}
\proof{این قضیه \lr{Ghouila-Houri} نام دارد که در کلاس و در جزوه اثبات شده است.}
\newline
\newline
\newline
حال به اثبات قضیه‌ی ۲ بر می‌گردیم، ماتریس $A$ یک ماتریس یک‌ متوالی سطری است.

%، بنابراین $A^T$ یک ماتریس یک متوالی ستونی است (به اختصار می‌گوییم یک متوالی)، و چون برای اثبات تک‌‌پیمانه‌ای بودن $A$، با دترمینان کار می‌کنیم و دترمینان هر ماتریس با ترانهاده‌اش برابر است، از این به بعد با همین $A^T$ کار می‌کنیم.
اثبات می‌کنیم که برای هر هر زیرمجموعه از ستون‌های $A$ مانند $S$، یک دو رنگ‌آمیزی متساوی برای $S$ موجود است.
 این حکم با اثبات دو گزاره‌ی زیر به دست می‌آید:
\begin{enumerate}
	\item اگر $A_S$را ماتریس تشکیل شده از انتخاب ستون‌های $S$ از $A$ درنظر بگیریم، آنگار $A_S$ یک ماتریس یک متوالی سطری است. اثبات این ادعا واضح است(در واقع اگر $A_S$ یک متوالی نباشد، می‌توانیم با افزایش آن به $A$، ببینیم که $A$ نیز یک متوالی نیست که تناقض است).
	\item هر ماتریس یک متوالی مانند $A$ دارای دو رنگ‌آمیزی متساوی روی ستون‌ها است.
\proof
ستون‌ها را یکی در میان آبی و قرمز می‌کنیم، حال می‌خواهیم تفاصل مجموع ستون‌های قرمز و مجموع ستون‌های آبی را به دست آوریم. برای سطر دلخواه $i$، درایه‌های  ستون‌های $l_i$ تا $r_i$ از این سطر برابر با ۱ هستند و بقیه‌ی درایه‌ها ۰ هستند. پس مجموع ستون‌های آبی منهای ستون‌های قرمز برای سطر $i$ام، برابر است با 
$\sum_{l=l_i}^{r_i} (-1)^l$ که یکی از اعداد $0$ یا $-1$ یا $1$ خواهد بود.
\end{enumerate}
با استفاده از گزاره‌ی (۱)، هر زیرمجموعه‌ از ستون‌ها مثل $S$ یک ماتریس یک متوالی می‌سازد و از گزاره‌ی (۲) نتیجه‌ می‌گیریم که این زیرماتریس $A_S$، رنگ آمیزی متساوی روی ستون‌ها دارد، پس طبق قضیه‌ی (۴)، $A$  تماما تک پیمانه‌ای است و حکم سوال اثبات می‌شود.
\begin{latin}
	$\square$
\end{latin}
\paragraph{ج-}
چندوجهی ساخته شده در بخش الف را با $P$ نشان می‌دهیم. طبق بخش  (ب)، تمام رئوس این چندوجهی صحیح می‌باشند و درنتیجه کافیست یک راس از این چندوجهی که جواب بهینه تولید می‌کند را پیدا کنیم. برای این کار، فرض می‌کنیم الگوریتمی داریم که با داشتن بردار $c$ و قیود چندوجهی $P$، یک جواب شدنی که حاصل
$c^Tx$
را بیشینه می‌کند پیدا می‌کند. نام این الگوریتم را $A$ می‌نامیم (که می‌تواند الگوریتمی مانند simplex یا الگوریتم‌های مشابه باشد). به وضوح نقطه‌ای که این الگوریتم پیدا می‌کند بر روی یکی از وجه‌های $P$ قرار دارد. در ابتدا چند وجهی $P$ قیودی معادل با چندوجهی آمده در بخش الف دارد و $c$ برداری است که وزن‌های $w_i$ را نشان می‌دهد (یعنی $c_i = w_i$).‌
با استفاده از الگوریتم $A$ می‌توانیم یک نقطه‌ی بهینه در چندوجهی $P$ پیدا کنیم و آن را با $x^*$ نشان دهیم. اگر $x^*$ صحیح بود که همان جواب خواسته شده‌ است. اگر صحیح نبود، اندیسی مانند $i$ وجود دارد که $x^*_i$ صحیح نیست. با توجه به اینکه $x^*$ روی یک وجه قرار دارد و با توجه به اینکه تمام رئوس چندوجهی به دلیل TU بودن صحیح می‌باشند، به ازای هر $k \in \{0,1\}$ جواب بهینه‌ی دیگری مانند $y^*$ وجود دارد که $y^*_i = k$. بنابراین کافیست به دلخواه یکی از $k=0,1$ را درنظر بگیریم و قید
$x_i = k$
را به قیود $P$ اضافه کنیم. همچنان قیودمان شرط TU بودن را دارند و بنابراین می‌توان همین الگوریتم را تکرار کرد تا زمانی که به یک جواب صحیح برسیم (با توجه به اینکه در مجموع m اندیس $x_i$ وجود دارد و اینکه در هر مرحله یک قید اضافه می‌شود که برای یکی از این اندیس‌ها شرط مساوی را اجبار می‌کند، در نهایت پس از حداکثر $m$ مرحله این الگوریتم متوقف می‌شود). اگر زمان اجرای الگوریتم $A$ را با $O(A)$ نشان دهیم، زمان اجرای کل الگوریتم از $O(Am)$ است. بنابراین یک الگوریتم چند جمله‌ای برای حل سوال خواسته شده پیدا کردیم.
\begin{latin}
	$\square$
\end{latin}
\end{document}
