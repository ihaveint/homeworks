\documentclass[a4paper,12pt]{article}
\usepackage{HomeWorkTemplate}
\usepackage{circuitikz}
\usepackage[shortlabels]{enumitem}
\usepackage{hyperref}
\usepackage{tikz}
\usepackage{listings}

\usepackage[utf8]{inputenc}
\usepackage[english]{babel}

\usepackage{hyperref}
\hypersetup{
    colorlinks=true,
    linkcolor=blue,
    filecolor=magenta,      
    urlcolor=cyan,
}

\urlstyle{same}


\usetikzlibrary{shapes.multipart}

\usepackage{amsmath}
\usepackage{amssymb}
\usepackage{tcolorbox}
\usepackage{xepersian}
\settextfont{XB Niloofar}
\usetikzlibrary{arrows,automata}
\usetikzlibrary{circuits.logic.US}
\usepackage{changepage}
\newcounter{problemcounter}
\newcounter{subproblemcounter}
\setcounter{problemcounter}{1}
\setcounter{subproblemcounter}{1}
\newcommand{\problem}[1]
{
	\subsection*{
		پرسش
%		\arabic{problemcounter} 
%		\stepcounter{problemcounter}
%		\setcounter{subproblemcounter}{1}
		#1
	}
}
\newcommand{\subproblem}{
	\textbf{\harfi{subproblemcounter})}\stepcounter{subproblemcounter}
}


\begin{document}
\handout
{بهینه‌سازی ترکیبیاتی مقدماتی}
{مرتضی علیمی، هانی احمد زاده}
{بهار 1400}
{اطلاعیه}
{سروش زارع}
{97100405}
 {تمرین تحویلی 2}

\paragraph{\color{red}برای سادگی در نوشتن، به ازای هر گراف $G$ ماتریس گفته شده در صورت سوال را با
$M(G)$
نشان می‌دهیم. 
همچنین به ازای یک سطر $i$ از $M(G)$ و یک رنگ آمیزی قرمز و آبی برای زیرمجموعه‌ی $C$ از ستون‌های
$M(G)$، 
$Sub(C, G)_i$
را برابر با درایه‌ی $i$ ام جمع‌ ستون‌های قرمز منهای جمع ستون‌های آبی به ازای این رنگ‌آمیزی می‌گیریم.
}
\problem{3}
\subproblem
طبق قضیه‌ی 
\lr{Ghouila-Houri}
نتیجه می‌گیریم که تماما تک‌پیمانه‌ای بودن گراف $G$ معادل این است که برای هر زیرمجموعه از ستون‌های $M(G)$،  یک دو رنگ آمیزی متساوی وجود داشته باشد. بنابراین فرض می‌کنیم که برای گراف $G$ این خاصیت برقرار است و می‌خواهیم نشان دهیم برای $H$ که یک زیرگراف القایی دلخواه از $G$ است نیز برقرار است. 
به ازای هر زیرمجموعه‌ی $C$ از ستون‌های $M(G)$، دو رنگ آمیزی متساوی متناظر آن را با 
\lr{coloring(C, G)}
نشان می‌دهیم. توجه کنید که تمام ستون‌های $M(H)$ در $M(G)$ وجود دارند و صرفا ممکن است در سطر‌ها این اتفاق نیفتد. حال ادعا می‌کنیم که به ازای هر زیرمجموعه‌ی ستون‌ها مانند $C'$ از $H$، همان رنگ آمیزی
\lr{S := coloring(C', G)}
برای
$H$
نیز دورنگ آمیزی متساوی است.
\newline
\proof{}
نشان می‌دهیم برای هر سطر $i$ در $M(H)$ که در تناظر با یک خوشه‌ی ماکسیمال $D$ در $H$ است، داریم: 
$Sub(C, H)_i \in \{-1,0,1\}$
(به ازای رنگ‌آمیزی $S$ که بالاتر تعریف شده است).
دو حالت برای $D$ درنظر می‌گیریم:
\begin{itemize}
	\item $D$ در گراف $G$ نیز یک خوشه‌ی ماکسیمال باشد \newline \newline
در این حالت، چون رنگ آمیزی $S$ دومتساوی است، سطر $i$ در $M(H)$ دقیقا متناظر با یک سطر $j$ در $M(G)$ است و درنتیجه داریم:
\begin{align*}
	\exists j: Sub(C, H)_i = Sub(C, G)_j \in \{-1,0,1\}
\end{align*}
از مجموع این دو حالت نتیجه می‌گیریم که اگر تعداد سطر‌های $M(H)$ برابر با $k$ باشد، داریم:
\begin{align*}
	\forall_{i \leq k} \ \ Sub(C, H)_i \in \{-1,0,1\}
\end{align*}
و درنتیجه رنگ‌آمیزی دومتساوی $S$ برای هر زیرمجموعه‌ی $C'$ از ستون‌های $G$، برای $H$  دومتساوی است و حکم سوال اثبات می‌شود.

بخش آخر از این مورد نتیجه شد که $S$ برای $M(G)$ یک رنگ آمیزی دو متساوی فرض شده است.
	\item $D$ در $G$ خوشه‌ی ماکسیمال نباشد و درنتیجه زیر مجموعه‌ی اکید یک خوشه‌ی ماکسمال $D'$ عضو $G$ باشد. \newline \newline
در این حالت چالش اصلی این است که سطری متناظر با سطر $i$ ام در $M(H)$ در $M(G)$ وجود ندارد. اما کافیست توجه کنیم که اگر $D'$ متناظر با سطر $j$ ام در $M(G)$ باشد، داریم:
\begin{align*}
	Sub(C, H)_i = Sub(C, G)_j \in \{-1,0,1\}
\end{align*}
\end{itemize}
پس اثبات کردیم که به ازای هر زیرمجموعه‌ی $C'$ از ستون‌های $H$، همان رنگ آمیزی
\lr{coloring(C', G)}
برای $H$ نیز دومتساوی است و درنتیجه حکم سوال اثبات می‌شود.
\begin{latin}
	$\square$
\end{latin} 
\subproblem
دور پنج‌تایی یعنی $C_5$ را درنظر بگیرید. در این گراف خوشه‌های ماکسیمال شامل تک یال‌های این گراف هستند و بنابراین ماتریس گفته شده در صورت سوال برای $C_5$ دقیقا معادل با ماتریس وقوع $C_5$ است. از طرفی طبق قضایای کتاب می‌دانیم که ماتریس وقوع یه گراف غیر جهت دار $G$ تماما تک پیمانه‌ای است اگر و تنها اگر $G$ دو بخشی باشد.  از آنجایی که $C_5$ دور فرد دارد، در نتیجه دو بخشی نیست و درنتیجه تماما تک پیمانه‌ای نیست که همان خواسته‌ی سوال است.
\begin{latin}
	$\square$
\end{latin} 
\subproblem
می‌دانیم که ماتریس وقوع گراف‌های دو بخشی $TU$ است. از طرفی در گراف دو بخشی دور فرد نداریم و درنتیجه تمام خوشه‌های ماکسیمال دقیقا متناظر با یک یال از گراف هستند. بنابراین ماتریس بررسی شده در صورت سوال، برای گراف‌های ۲ بخشی دقیقا همان ماتریس وقوع است و درنتیجه $TU$ است.
\begin{latin}
	$\square$
\end{latin} 

\subproblem
\theorem{ اگر یک گراف $G$ بازه‌ای باشد، یک ترتیب
\begin{align*}
	M_1 , ... M_k
\end{align*}
برای خوشه‌های ماکسیمال  $G$ وجود دارد به طوری که به ازای هر راس $v \in V(G)$، خوشه‌های ماکسیمالی که $v$ عضو آن‌هاست، یک زیر دنباله‌ی متوالی از ترتیب گفته شده بسازند.
}
\proof{}
یک خوشه‌ی ماکسیمال از گراف بازه‌ای $G$ را درنظر بگیرید. می‌توان دید که یک بازه‌ی 
$[l, u)$
وجود دارد که اکیدا داخل تمام بازه‌های متناظر با راس‌های این خوشه قرار دارد. این بازه‌ی را به عنوان نماینده‌ای از این خوشه درنظر می‌گیریم. همچنین می‌توان دید که اگر نماینده‌های تمام خوشه‌های ماکسیمال را درنظر بگیریم، دو به دو اشتراکشان تهی است (در غیر این صورت دو خوشه‌ی $M_1$ و $M_2$ که نماینده‌هایشان اشتراک دارند، در واقع زیرمجموعه‌ی اکید خوشه‌ی $M_1 \cup M_2$ می‌باشند که با ماکسیمال بودنشان در تناقض است). پس می‌توانیم خوشه‌ها را به ترتیب
$M_1$, ... $M_k$
مرتب کنیم به طوری که اگر نماینده‌ی $M_i$ را با
$[l_i, r_i)$
نشان دهیم، داشته باشیم:
\begin{align}
	l_1 < r_1 \leq l_2 < r_2 ... \leq l_k < r_k
\end{align}
حال ادعا می‌کنیم که همین چینش از خوشه‌ها، دقیقا خاصیت گفته شده در قضیه (۱) را دارد. فرض کنید این طور نباشد (برهان خلف)، بنابراین به ازای یک راس $v \in V(G)$، سه اندیس
$i < j < t$
وجود دارد به طوری که
\begin{align*}
v \in M(i), v \notin M(j) , v \in M(t)	
\end{align*}
پس اگر بازه‌ی متناظر $v$ را با $I_v := [l_v, r_v)$ نشان دهیم، :
\begin{align*}
	l_v \leq l_i \quad &and \quad r_v > r_i \\
	l_v > l_j \quad &or \quad r_v \leq r_j \\
	l_v \geq l_t \quad &and \quad r_v > r_t
\end{align*}
 با لحاظ کردن نامعادلات (۱) داریم:
\begin{align*}
	l_v \leq l_i < l_j \quad &and \quad r_v > r_t > r_j \\
	l_v > l_j \quad &or \quad r_v \leq r_j
\end{align*}
که به وضوح تناقض است. در نتیجه فرض خلف باطل است و حکم قضیه‌ی ۱ اثبات می‌شود.


بنابر قضیه‌ی (۱)، ماتریس $M(G)$، ماتریسی 
\underline{یک‌ متوالی} است  که در سوال ۲ و همچنین در جزوه اثبات شده است چنین ماتریسی تماما تک پیمانه‌ای است، در نتیجه حکم سوال اثبات می‌شود.
\begin{latin}
	$\square$
\end{latin} 
\end{document}
