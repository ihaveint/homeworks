\documentclass[a4paper,12pt]{article}
\usepackage{HomeWorkTemplate}
\usepackage{circuitikz}
\usepackage[shortlabels]{enumitem}
\usepackage{hyperref}
\usepackage{tikz}
\usepackage{listings}

\def\Max{\text{بیشینه کن}}
\def\Min{\text{کمینه کن}}
\def\st{\text{\rl{که}}}

\usetikzlibrary{shapes.multipart}

\usepackage{amsmath}
\usepackage{amssymb}
\usepackage{tcolorbox}
\usepackage{xepersian}
\settextfont{XB Niloofar}
\usetikzlibrary{arrows,automata}
\usetikzlibrary{circuits.logic.US}
\usepackage{changepage}
\newcounter{problemcounter}
\newcounter{subproblemcounter}
\setcounter{problemcounter}{1}
\setcounter{subproblemcounter}{1}
\newcommand{\problem}[1]
{
	\subsection*{
		پرسش
%		\arabic{problemcounter} 
%		\stepcounter{problemcounter}
%		\setcounter{subproblemcounter}{1}
		#1
	}
}
\newcommand{\subproblem}{
	\textbf{\harfi{subproblemcounter})}\stepcounter{subproblemcounter}
}


\begin{document}
\handout
{بهینه‌سازی ترکیبیاتی مقدماتی}
{مرتضی علیمی، هانی احمد زاده}
{بهار 1400}
{اطلاعیه}
{سروش زارع}
{97100405}
 {تمرین تحویلی 2}
\problem{1}
\subproblem
 اگر $\chi^R$ بردار مشخصه‌ی $R$ باشد، داریم:
\begin{align*}
	|R \cap \{\sigma_i(2t-1), \sigma_i(2t) \} | = \chi^R_{\sigma_i(2t-1)} + \chi^R_{\sigma_i(2t)} = 1
\end{align*}
بنابراین همین قید‌ها را می‌توانی برای $x$ های شدنی لحاظ کنیم و به چندوجهی زیر برسیم:
\begin{align*}
\quad  Ax &\leq 1\\
\quad  -Ax &\leq -1\\
\quad 0 &\leq x_i \leq 1 \quad \forall i \in [n]
\end{align*}
که ماتریس $A$ اینگونه تعریف می‌شود که $n/2$ سطر اول آن متناظر با $n/2$ قید به شکل زیر
\begin{align*}
	x_{\sigma_1(2t - 1)} + x_{\sigma_1(2t)} \ \ \forall t \in [n/2]
\end{align*}
و $n/2$ سطر بعدی متناظر با قید‌های
\begin{align*}
	x_{\sigma_2(2t - 1)} + x_{\sigma_2(2t)} \ \ \forall t \in [n/2]
\end{align*}
می‌باشند، یعنی در هر کدام از این سطر‌ها دو درایه  $2t-1$ و $2t$ برابر با ۱ هستند و بقیه‌ی درایه‌ها ۰ هستند.
اگر چندوجهی حاصل از این قیود را با $Q$ نشان دهیم، ادعا می‌کنیم 
$Q = P_R$
\proof{}
ابتدا نشان می‌دهیم هر راس  $v$ از $P_R$ عضو $Q$ است. این مورد تقریبا واضح است، زیرا رئوس $P_R$ را دقیقا بردار مشخصه‌هایی درنظر گرفتیم که شروط گفته شده مربوط به زیرمجموعه‌ی خوب برایشان برقرار بود، که عملا معادل با همان قیودی است که در چندوجهی $Q$ آورده‌ایم. بنابراین  $v \in Q$.
از آنجایی که تمام رئوس $P_R$ عضو $Q$ هستند و $Q$ و $P_R$ هر دو مجموعه‌ی محدب هستند، $P_R \subseteq Q$.
\newline
حال نشان می‌دهیم که تمام رئوس $Q$ نیز داخل $P_R$ هستند. برای این مورد  نشان می‌دهیم هر راس $Q$ صحیح است و درنتیجه  هر راس از $Q$ دقیقا متناظر با یکی از راس‌های $P_R$ است. دلیل این اتفاق این است که اگر به قید‌های $Q$ نگاه کنیم، می‌بینیم که $n/2$ سطر اول آن، دقیقا مانند ماتریس وقوع در یک گراف دو بخشی می‌باشند و $n/2$ سطر دوم آن نیز ماتریس وقوع در یک گراف دو بخشی دیگر می‌باشند، و هدف پیدا کردن یک تطابق است. بنابراین در عمل انگار یک گراف دو بخشی داریم و سعی می‌کنیم که یک تطابق از آن را پیدا کنیم. از آنجایی که ماتریس  وقوع گراف‌های دو بخشی $TU$ است، تمام راس‌های $Q$ صحیح می‌باشند و درنتیجه هر راس از $Q$ داخل $P_R$ است. به دلیل محدب بودن $Q$ و $P_R$، نتیجه می‌گیریم که
$Q \subseteq P_R$.
از دو گزاره‌ی اثبات شده نتیجه می‌شود که
$Q = P_R$
که همان حکم سوال است.
%\newline
\begin{latin}
	$\square$
\end{latin}
\subproblem
در قسمت (الف) دیدیم که نمایش چند وجهی گفته شده به صورت نامساوی‌ها به صورت 
$Ax \leq b$
قابل بیان است که $A$ ماتریسی $TU$ است. از طرفی با قرار دادن بردار $x$ به طوری که همه‌ی $x_i$ ها برابر با $1/2$ باشند، به وضوح می‌بینیم که شرط بخش (الف) برقرار است و درنتیجه چند وجهی زیرمجموعه‌های خوب ناتهی است. از طرفی به خاطر $TU$ بودن ماتریس  $A$، تمام راس‌های این چند وجهی صحیح هستند و درنتیجه حداقل یک جواب بهینه‌ی صحیح دارد. اگر زیرمجموعه‌ی $R$ متناظر با این جواب بهینه را با $R^*$ نشان دهیم، ادعا می‌کنیم که $R^*$ یک رنگ آمیزی با اختلاف ۲ به ما می‌دهد.

برهان:
کافی است اثبات کنیم به ازای هر
$i = 1 , 2$
و هر
$l, u \in [n]$
که
$l < u$
شروط گفته شده در ابتدای سوال برقرارند.
کافی است اثبات کنیم به ازای $i \in \{0,1\}$ دلخواه و دو مقدار
$l , u \in [n]$
دلخواه که
$l < u$
این شروط برقرارند.
داریم:
$I := \{\sigma_i(l), \sigma_i(l+1), ..., \sigma_i(u) \}$
حال سعی می‌کنیم همین $I$ را بازنویسی کنیم. ابتدا تعریف می‌کنیم:
\begin{align*}
\forall t \in [n/2] : D(i, t) := \{\sigma_i(2t-1), \sigma_i(2t) \}	
\end{align*}
حال داریم:
\begin{align}
	I = [\bigcup_{j=\lceil (l+1)/2 \rceil}^{\lfloor r/2 \rfloor} D(i,j)] \cup L \cup U
\end{align}
به طوری که داریم:
\begin{align*}
	L =
\left\{
	\begin{array}{ll}
		\{\sigma_i (l)\}  & \mbox{if } l \ mod \ 2 = 0\\
		\phi & \mbox{else }
	\end{array}
\right. \\ \\
	U =
\left\{
	\begin{array}{ll}
		\{\sigma_i (u)\}  & \mbox{if } u \ mod \ 2 = 1\\
		\phi & \mbox{else }
	\end{array}
\right.
\end{align*}

در واقع بازه‌ی $I$ را به تعدادی بازه‌ به شکل
$\{\sigma_i(2t-1), \sigma_i(2t) \}$
(و در صورت نیاز برای ابتدا و انتهای $I$، دو مجموعه‌ی تک عضوی $L$ و $U$) افراز کرده‌ایم. حال  با درنظر گرفتن $R^*$ می‌بینیم که به ازای هر کدام از این مجموعه‌های ۲ عضوی که بخشی از $I$ را ساخته‌اند، 
$|R \cap \{\sigma_i(2t-1), \sigma_i(2t) \}| = 1$
برقرار است و همچنین به طور مشابه داریم:
\begin{align*}
|\{\sigma_i(2t-1), \sigma_i(2t) \} \backslash R| = 1	
\end{align*}

و درنتیجه داریم:
\begin{align*}
	||\{\sigma_i(2t-1), \sigma_i(2t) \} \cap R| - |\{\sigma_i(2t-1), \sigma_i(2t) \} \backslash R|| = 0
\end{align*}
با جمع زدن این تساوی روی تمام زیرمجموعه‌های ۲ تایی در (۱)، می‌بینیم که بدون درنظر گرفتن $L$ و $U$، این تساوی همچنان صفر باقی می‌ماند. در نهایت با اضافه کردن $L$ و $U$ می‌بینیم که ممکن است دیگر تساوی برقرار نباشد ولی از آنجایی که هر کدام از $L$ و $U$ یک عضو دارند، حداکثر به اندازه‌ی ۲ یا منفی ۲ از صفر فاصله می‌گیریم، پس همواره شرط
\begin{align*}
	||I \cap R^*| - |I \backslash R^*|| \leq 2
\end{align*}
برقرار است. در نتیجه $R^*$ متناظر با یک رنگ‌آمیزی با اختلاف ۲ است و حکم این قسمت اثبات می‌شود.
\begin{latin}
	$\square$
\end{latin}
\end{document}
