\documentclass[a4paper,12pt]{article}
\usepackage{HomeWorkTemplate}
\usepackage{circuitikz}
\usepackage[shortlabels]{enumitem}
\usepackage{hyperref}
\usepackage{tikz}
\usepackage{listings}

\def\Max{\text{بیشینه کن}}
\def\Min{\text{کمینه کن}}
\def\st{\text{\rl{که}}}


\usetikzlibrary{shapes.multipart}

\usepackage{amsmath}
\usepackage{amssymb}
\usepackage{tcolorbox}
\usepackage{xepersian}
\settextfont{XB Niloofar}
\usetikzlibrary{arrows,automata}
\usetikzlibrary{circuits.logic.US}
\usepackage{changepage}
\newcounter{problemcounter}
\newcounter{subproblemcounter}
\setcounter{problemcounter}{1}
\setcounter{subproblemcounter}{1}
\newcommand{\problem}[1]
{
	\subsection*{
		پرسش
%		\arabic{problemcounter} 
%		\stepcounter{problemcounter}
%		\setcounter{subproblemcounter}{1}
		#1
	}
}
\newcommand{\subproblem}{
	\textbf{\harfi{subproblemcounter})}\stepcounter{subproblemcounter}
}


\begin{document}
\handout
{بهینه‌سازی ترکیبیاتی مقدماتی}
{مرتضی علیمی، هانی احمد زاده}
{بهار 1400}
{اطلاعیه}
{سروش زارع}
{97100405}
 {تمرین تحویلی 3}
\problem{4}
سیستم $Q_1$ را به صورت زیردرنظر می‌گیریم:
\begin{align*}
	A_1x \leq b_1 \\
	A_2x \leq b_2
\end{align*}
به طور مشابه سیستم $Q_2$ را به صورت زیردرنظر می‌گیریم:
\begin{align*}
	A_1'x \leq b_1' \\
	A_2'x \leq b_2
\end{align*}
باید اثبات کنیم که دستگاه $Q_2$ تمام دوگان صحیح است. بنابراین به ازای یک $c$ که $Q_2$ و دوگان آن جواب بهینه‌ دارند، باید اثبات کنیم که برنامه‌ی دوگان $Q_2$ یعنی 
\begin{align*}
\text{\Min} &\quad [b_1'^T \ \ b_2^T] y\\
\text{\st} &\quad  [A_1'^T \ \ A_2^T]y = c\\
&\quad y \geq 0
\end{align*}
که آن را با $D(Q_2)$ نشان می‌دهیم، جواب بهینه‌ی صحیح دارد.
حال دستگاه زیر را درنظر بگیرید:
\begin{align*}
\text{\Min} &\quad [b_1^T \ \ b_2^T] y\\
\text{\st} &\quad  [A_1^T \ \ A_2^T]y = c\\
&\quad y \geq 0
\end{align*}
این دستگاه دوگان $Q_1$ است و آن را با $D(Q_1)$ نشان می‌دهیم. با توجه به تماما دوگان صحیح بودن $Q_1$، این دستگاه جواب بهینه‌ی صحیحی مانند
\begin{align}
y^* = \begin{pmatrix} y_1^* \\ y_2^* \end{pmatrix}	
\end{align}
\begin{latin}
	$\square$
\end{latin}
دارد به طوری که
\begin{align*}
	b_2^Ty_2^* = m_2 \\
	b_1^Ty_1^* = m_1 \\
	A_2^Ty_2^* = c_2 \\
	A_1^Ty_1^* = c_1
\end{align*}
که $c = c_1 + c_2$ و 
مقدار
$m := m_1 + m_2$
همان مقدار بهینه‌ی دستگاه است.
با توجه به  تساوی
\begin{align}
	\{x | A_1'x \leq b_1' \}	 = \{x | A_1x \leq b_1 \}	
\end{align}
فضای شدنی دوچند وجهی
$Q_1$
و $Q_2$
برابرند و درنتیجه جواب بهینه‌ی آن‌ها نیز برابر است.
حال نشان می‌دهیم بردار‌های صحیح
$y_1, y_2 \geq 0$
وجود دارند که 
\begin{align}
b_1'^Ty_1 + b_2^Ty_2 = m \\
A_1'^Ty_1 + A_2^Ty_2 = c
\end{align}

طبیعتا می‌توانیم قرار دهیم
$y_2 := y_2^*$
و در ادامه سعی می‌کنیم مقدار مناسبی برای $y_1$ پیدا کنیم.
\newline
دستگاه زیر را درنظر بگیرید:
\begin{align*}
\text{\Min} &\quad b_1 y\\
\text{\st} &\quad  A_1^T y_1 = c_1\\
&\quad y_1 \geq 0
\end{align*}
مقدار جواب بهینه‌ی این دستگاه $m_1$ است (در غیر این‌صورت اگر این جواب بهینه توسط بردار $y''$ تولید شود، با قرار دادن $y''$ به جای $y_1^*$  و تغییر ندادن $y_2^*$ جواب بهتری از (۱)  خواهیم داشت که با بهینه بودن (۱) در تناقض است).  طبق قضیه‌ی دوگانی قوی، مقدار جواب بهینه‌ی دوگان این دستگاه نیز  $m_1$ است.
دوگان این دستگاه را می‌نویسیم:
\begin{align*}
\text{\Max} &\quad c_1^T x\\
\text{\st} &\quad  A_1x \leq b_1
\end{align*}
با توجه به تساوی (۲)، این دوگان را به شکل زیر می‌نویسیم
\begin{align*}
\text{\Max} &\quad c_1^T x\\
\text{\st} &\quad  A_1'x \leq b_1'
\end{align*}
طبق فرض سوال، این دستگاه تماما دوگان صحیح است. پس اگر به ازای یک $c$ صحیح این دستگاه و دوگان آن جواب بهینه داشته باشند، دوگان آن جواب بهینه‌ی صحیح دارد.
حال دوگان دستگاه بالا را می‌نویسیم:
\begin{align*}
\text{\Min} &\quad  y_1^T b_1'\\
\text{\st} &\quad  A_1'^T y_1 = c_1 \\
&\quad y_1 \geq 0
\end{align*}
با توجه به اینکه خود دستگاهمان مقدار جواب بهینه‌ی $m_1$ داشت، دوگان آن نیز طبق قضیه‌ی دوگانی قوی جواب بهینه‌ی $m_1$ دارد و با توجه به تماما دوگان صحیح بودن دستگاه اصلی، دوگان جواب بهینه‌ی صحیح دارد. از طرفی اگر این جواب بهینه‌ی صحیح را با $y''$ نشان دهیم، می‌توانیم قرار دهیم 
$y_1 = y''$
(از قبل قرارداده بودیم $y_2 = y_2^*$) به این ترتیب تساوی‌های (۳) و (۴) برقرار می‌شوند و $y_1$ و $y_2$ نیز بردار‌های صحیح نامنفی هستند. پس دستگاه $Q_2$ تماما دوگان صحیح است و حکم اثبات می‌شود.
\begin{latin}
	$\square$
\end{latin}

\end{document}
