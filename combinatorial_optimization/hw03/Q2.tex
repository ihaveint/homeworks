\documentclass[a4paper,12pt]{article}
\usepackage{HomeWorkTemplate}
\def\Max{\text{بیشینه کن}}
\def\Min{\text{کمینه کن}}
\def\st{\text{\rl{که}}}

\usepackage{circuitikz}
\usepackage[shortlabels]{enumitem}
\usepackage{hyperref}
\usepackage{tikz}
\usepackage{listings}

\usetikzlibrary{shapes.multipart}

\usepackage{amsmath}
\usepackage{amssymb}
\usepackage{tcolorbox}
\usepackage{xepersian}
\settextfont{XB Niloofar}
\usetikzlibrary{arrows,automata}
\usetikzlibrary{circuits.logic.US}
\usepackage{changepage}
\newcounter{problemcounter}
\newcounter{subproblemcounter}
\setcounter{problemcounter}{1}
\setcounter{subproblemcounter}{1}
\newcommand{\problem}[1]
{
	\subsection*{
		پرسش
%		\arabic{problemcounter} 
%		\stepcounter{problemcounter}
%		\setcounter{subproblemcounter}{1}
		#1
	}
}
\newcommand{\subproblem}{
	\textbf{\harfi{subproblemcounter})}\stepcounter{subproblemcounter}
}


\begin{document}
\handout
{بهینه‌سازی ترکیبیاتی مقدماتی}
{مرتضی علیمی، هانی احمد زاده}
{بهار 1400}
{اطلاعیه}
{سروش زارع}
{97100405}
 {تمرین تحویلی 3}
 تعریف می‌کنیم
\begin{align*}
P' := CH(\{\chi^F | F \in \mathcal{F} \}
\end{align*}

\paragraph{ابتدا نشان می‌دهیم که $P' \subseteq P$.}
اثبات این مورد واضح است، با توجه به محدب بودن $P$ و $P'$ کافی است نشان دهیم تمام راس‌های $P'$ در $P$ قرار دارند. از طرفی به ازای هر 
$\chi^F | F \in \mathcal{F}$
 به وضوح تمام شروط 
\begin{align*}
	\sum_{I \in \mathcal{I} : e \in I} x(I) \geq 1, \forall e \in E
\end{align*}
برقرار است (با توجه به تعریف $\mathcal{F}$). پس تمام رئوس $P'$ در $P$ قرار دارند و درنتیجه
\begin{align}
	P' \subseteq P
\end{align}
\paragraph{حال نشان می‌دهیم که $P \subseteq P'$.}
برای این کار نیز نشان می‌دهیم که ماتریس توصیف کننده‌ی قیود $P$ تماما تک پیمانه‌ای است و درنتیجه تمام رئوس $P$ صحیح هستند و با توجه به اینکه فقط مقادیر 0 و 1 برای درایه‌های صحیح مجاز است (طبق تعریف $P$)، تمام رئوس $P$ در $P'$ قرار خواهند داشت.
ماتریس $A$ متناظر با قیود به شکل
\begin{align*}
	\sum_{I \in \mathcal{I} : e \in I} x(I) \geq 1, \forall e \in E
\end{align*}
را درنظر بگیرید.
اگر سطر‌های این ماتریس را متناظر با یال‌های 
$e \in E$
درنظر بگیریم و ستون‌ها را متناظر با
$I \in \mathcal{I} $
هر ستون متناظر یک زیرمسیر از مسیری مانند $P(v)$ است (طبق صورت سوال). این ماتریس را با $A$ نشان می‌دهیم. اثبات می‌کنیم که $A^T$ تماما تک پیمانه‌ای است و درنتیجه $A$ نیز تماما تک پیمانه‌ای است. 
\newline
ستون‌های $A^T$ معادل با یال‌های $e \in E$ هستند. کافی است طبق قضیه‌ی
$Ghouila-Houri$
نشان دهیم که برای هر زیرمجموعه از این ستون‌ها، یک رنگ آمیزی دو متساوی وجود دارد. این کار نیز به راحتی انجام می‌شود، کافی است برای هر زیرمجموعه‌ی 
$E' \subseteq E$
بردار مشخصه‌ی
$\chi^{E'}$
را درنظر بگیریم و به هر یال $e \in E$ وزن
$\chi^{E'}_{e}$
را نسبت دهیم. در واقع وزن هر یال ۱ است اگر و تنها اگر در $E'$ آمده باشد و درغیر اینصورت ۰ است.  این وزن را با $w(e)$ نشان می‌دهیم. همچنین به ازای هر یال $e$ مقدار $d(e)$ را به صورت بازگشتی تعریف می‌کنیم:
\begin{itemize}
	\item اگر $e=uv$ که $u$ ریشه‌ی درخت است، $d(e):=w(e)$
	\item اگر $e=uv$ که $u$ پدر $v$ است به طوری که پدر $u$ برابر با $u'$ است، تعریف می‌کنیم:
	\begin{align*}
		d(e) = w(e) + d(u'u)
	\end{align*}
\end{itemize}
در واقع $d(e)$ نشان دهنده‌ی جمع وزن‌ یال‌هایی است که از ریشه شروع شده‌اند و درنهایت با طی یک سری یال به $e$ رسیده‌اند  (خود $e$ را نیز درنظر می‌گیریم).
حال به ازای هر 
$e \in E | w(e) = 1$
بسته به اینکه
$d(e)$
زوج است یا فرد، رنگ ستون متناظر با $e$ در $E'$ را آبی و قرمز می‌کنیم. به راحتی می‌توان دید که با توجه به تعریف $d(e)$، به ازای هر سطر $A^T$  که متناظر با یکی از $I \in \mathcal{I}$ ها است، درایه‌ی مربوط به این سطر در اختلاف ستون‌های آبی و قرمز برابر با یکی از اعداد منفی ۱ و صفر و مثبت ۱ است (دلیل این اتفاق نیز این است که $I$ متناظر با یک زیر مسیر از یک $P(v)$ است و بنابراین با درنظر گرفتن ستون‌های $E'$ و رنگ آمیزی بیان شده، یال‌های $I$ (به جز یال‌هایی که وزن ۱ دارند که عملا از ستون‌ها حذف شده‌اند) یک در میان آبی و قرمز شده‌اند که نتیجه‌ی مدنظر را می‌دهد). بنابراین به ازای تمام سطر‌های $A^T$ این خاصیت برقرار است و رنگ آمیزی بیان شده یک رنگ آمیزی دو متساوی است. 
درنتیجه $A^T$ ماتریسی $TU$ است و به دنبال آن $A$ نیز $TU$ است.
پس تمام رئوس $P$ صحیح می‌باشند و درنتیجه داخل $P'$ خواهند بود و داریم:
\begin{align}
	P \subseteq P'
\end{align}
با توجه به نتایج (۱) و (۲) داریم
\begin{align*}
	P = P' = CH(\{\chi^F | F \in \mathcal{F} \}
\end{align*}
که همان حکم سوال است.
\begin{latin}
	$\square$
\end{latin}
\end{document}
