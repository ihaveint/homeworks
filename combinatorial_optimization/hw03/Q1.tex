\documentclass[a4paper,12pt]{article}
\usepackage{HomeWorkTemplate}
\usepackage{circuitikz}
\usepackage[shortlabels]{enumitem}
\usepackage{hyperref}
\usepackage{tikz}
\usepackage{listings}

\def\Max{\text{بیشینه کن}}
\def\Min{\text{کمینه کن}}
\def\st{\text{\rl{که}}}

\usetikzlibrary{shapes.multipart}

\usepackage{amsmath}
\usepackage{amssymb}
\usepackage{tcolorbox}
\usepackage{xepersian}
\settextfont{XB Niloofar}
\usetikzlibrary{arrows,automata}
\usetikzlibrary{circuits.logic.US}
\usepackage{changepage}
\newcounter{problemcounter}
\newcounter{subproblemcounter}
\setcounter{problemcounter}{1}
\setcounter{subproblemcounter}{1}
\newcommand{\problem}[1]
{
	\subsection*{
		پرسش
%		\arabic{problemcounter} 
%		\stepcounter{problemcounter}
%		\setcounter{subproblemcounter}{1}
		#1
	}
}
\newcommand{\subproblem}{
	\textbf{\harfi{subproblemcounter})}\stepcounter{subproblemcounter}
}


\begin{document}
\handout
{بهینه‌سازی ترکیبیاتی مقدماتی}
{مرتضی علیمی، هانی احمد زاده}
{بهار 1400}
{اطلاعیه}
{سروش زارع}
{97100405}
 {تمرین تحویلی 3}
\problem{1}
\lemma{ اگر به ازای یک $x \in P$ و $T,T' \subseteq S$ داشته باشیم 
\begin{align*}
	x(T) = r(T) \\
	x(T') = r(T')
\end{align*}
خواهیم داشت:
\begin{align*}
	x(T \cap T') = r(T \cap T') \\
	x(T \cup T') = r(T \cup T')
\end{align*}
}
\proof{
داریم:
\begin{align*}
	x(T\cap T') + x(T \cup T') \leq r(T \cap T') + r(T \cup T') \leq r(T) + r(T') = x(T) + x(T')
\end{align*}
نامساوی اول به دلیل قیود $P$ است، نامساوی دوم به دلیل فرض (ب) در صورت سوال است و تساوی آخر به دلیل فرض لم است. 
حال توجه کنید که طرف راست و چپ این عبارت با هم برابرند. بنابراین تمام نامساوی ها به صورت تساوی برقرارند و به طور دقیق تر داریم:
\begin{align*}
	x(T \cap T') = r(T \cap T') \\
	x(T \cup T') = r(T \cup T')
\end{align*}
که همان لم خواسته شده است.
\begin{latin}
	$\square$
\end{latin}
حال نقطه‌ی گوشه‌ای $x$ با شروط
\begin{align*}
	x_e >0, \forall e \in S
\end{align*}
درنظر می‌گیریم. با توجه به اینکه $x$ گوشه‌ای است، باید در $|S|$ قید مستقل خطی به شکل
$x(T) = r(T)$
صدق کند. برای هر کدوم از این قیود $\chi^T$ را درنظر می‌گیریم و مجموعه‌ی این $\chi^T$ ها را به صورت سطر به سطر در یک ماتریس $A$ قرار می‌دهیم. همچنین این سطر‌ها را مرتب می‌کنیم به طوری که تعداد ۱ ها در سطر‌ها از بالا به پایین نزولی باشد.
ماتریس $A$ ماتریسی $|S| * |S|$ است. با توجه به اینکه سطر‌ها را مستقل خطی درنظر گرفتیم، رنک سطری برابر با $|S|$ است و درنتیجه رنک ستونی نیز $|S|$ است. حال سطر اول را درنظر بگیرید، اگر در تمام درایه‌های $i$ که این سطر برابر با ۱ است، سطر دوم نیز برابر با ۱ باشد، با توجه به اینکه سطر‌ها را به ترتیب نزولی چیده‌ایم، سطر اول و دوم دقیقا برابرند و مستقل خطی نمی‌باشند که تناقص است. پس سطر دوم حداقل در یکی از درایه‌هایی که سطر اول مقدار ۱ دارد، مقدار ۰ دارد. به طور مشابه سطر سوم حداقل در یکی از درایه‌هایی که سطر دوم برابر با ۱ است، مقدار ۰ دارد وهمین استدلال را تا انتها می‌توانیم ادامه دهیم.
پس می‌توانیم با اشتراک گیری، به بردار‌هایی به شکل زیر برسیم:
\begin{align}
\begin{pmatrix} 1 \\ 0 \\ \vdots \\ 0 \\ 0 \end{pmatrix}	
,
\begin{pmatrix} ? \\ 1 \\ 0 \\ \vdots \\ 0 \end{pmatrix}	
, ... ,
\begin{pmatrix} ? \\ ? \\ \vdots \\ ? \\ 1 \end{pmatrix}	
\end{align}
که مقادیر علامت سوال می‌توانند هر کدام از مقدار‌های ۰ و ۱ باشند. نکته‌ی کلیدی این است که این بردارها از اشتراک گیری تعدادی بردار که متناظر با قیودی به شکل $x(T) = r(T)$ می‌باشند به دست آمده‌اند و درنتیجه طبق لم (۱)، هر کدام از بردار‌های $v$ در بالا، در شرط
$x(v) = r(v)$
صدق می‌کند. همچنین اگر از اجتماع گیری نیز استفاده کنیم (هر بردار را با بردارهای قبل خود اجتماع بگیریم، به بردار‌های زیر می‌رسیم:
\begin{align}
\begin{pmatrix} 1 \\ 0 \\ \vdots \\ 0 \\ 0 \end{pmatrix}	
,
\begin{pmatrix} 1 \\ 1 \\ 0 \\ \vdots \\ 0 \end{pmatrix}	
, ... ,
\begin{pmatrix} 1 \\ 1 \\ \vdots \\ 1 \\ 1 \end{pmatrix}	
\end{align}
این بردار‌ها به وضوح مستقل خطی هستند و همچنین مجددا طبق لم (۱) نتیجه می‌گیریم که به ازای هر بردار $v$ از بین این بردار‌ها، داریم
$x(v) = r(v)$.
بنابراین شرط (ب) خواسته شده در حکم سوال برقرار است. 
همچنین این بردارها به وضوح تشکیل یک زنجیر می‌دهند که تعداد اعضای آن $|S|$ است و درنتیجه شرط (ج) نیز برقرار است. شرط (آ) نیز به خاطر برقراری لم (۱)  و نحوه‌ی ساختن بردار‌های  (2) برقرار است.
پس تمامی شرط‌ها برقرار است و زنجیر مدنظر را ساختیم.
\begin{latin}
	$\square$
\end{latin}
\end{document}
