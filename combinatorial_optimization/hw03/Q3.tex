\documentclass[a4paper,12pt]{article}
\usepackage{HomeWorkTemplate}
\usepackage{circuitikz}
\usepackage[shortlabels]{enumitem}
\usepackage{hyperref}
\usepackage{tikz}
\usepackage{listings}

\usepackage[utf8]{inputenc}
\usepackage[english]{babel}

\usepackage{hyperref}
\hypersetup{
    colorlinks=true,
    linkcolor=blue,
    filecolor=magenta,      
    urlcolor=cyan,
}

\urlstyle{same}


\usetikzlibrary{shapes.multipart}

\usepackage{amsmath}
\usepackage{amssymb}
\usepackage{tcolorbox}
\usepackage{xepersian}
\settextfont{XB Niloofar}
\usetikzlibrary{arrows,automata}
\usetikzlibrary{circuits.logic.US}
\usepackage{changepage}
\newcounter{problemcounter}
\newcounter{subproblemcounter}
\setcounter{problemcounter}{1}
\setcounter{subproblemcounter}{1}
\newcommand{\problem}[1]
{
	\subsection*{
		پرسش
%		\arabic{problemcounter} 
%		\stepcounter{problemcounter}
%		\setcounter{subproblemcounter}{1}
		#1
	}
}
\newcommand{\subproblem}{
	\textbf{\harfi{subproblemcounter})}\stepcounter{subproblemcounter}
}


\begin{document}
\handout
{بهینه‌سازی ترکیبیاتی مقدماتی}
{مرتضی علیمی، هانی احمد زاده}
{بهار 1400}
{اطلاعیه}
{سروش زارع}
{97100405}
 {تمرین تحویلی 3}

\problem{3}
\subproblem
قضیه‌ی زیر در کلاس اثبات شده‌ است:
\theorem{فرض کنید $A$ ماتریسی $TU$ باشد. به ازای بردار $y$، بردار $z$ وجود دارد که گرد شده‌ی $y$ است و $Az$ گردشده‌ی $Ay$ است.}
پس تعمیم این قضیه به سوال مدنظر، کافیست ماتریس $A$ را بسازیم و نشان دهیم که $TU$ است. از طرفی به راحتی می‌توان دید که ماتریس $A$ یک ماتریس بازه‌ای است به این معنا که هر سطر آن تعدادی یک متوالی دارد (هر سطر متناظر با اندیس‌های یکی از جمع‌هایی به شکل $\sum_k^j$ است به این معنی که اندیس‌های $k$ تا $j$ آن ۱ است و بقیه‌ی اندیس‌ها ۰ هستند).  بنابراین ترانهاده‌ی $A$ یک ماتریس یک متوالی است و در جزوه (و همچنین تمرین سری ۲) اثبات شده است که ماتریس‌های یک متوالی $TU$ هستند. پس $A$ نیز $TU$ است و حکم سوال نتیجه می‌شود.
\begin{latin}
	$\square$
\end{latin}
\subproblem
در این بخش به جای استفاده از لم نامتناهی کونیگ، از حالت خاص آن برای درخت استفاده می‌کنیم. این حالت خاص بیان می‌کند که هر درخت نامتناهی یا راسی دارد که درجه‌ی نامتناهی دارد، یا مسیری نامتناهی دارد (همچنین می‌توان نشان داد در چنین حالتی مسیری نامتناهی داریم که یک سر آن ریشه‌ی درخت است که از این بیان استفاده می‌کنیم).
یک درخت $T$ تعریف می‌کنیم:
\begin{itemize}
	\item در ابتدا $T$ فقط یک ریشه دارد که روی آن عدد 0 نوشته شده است. به طور کلی عدد نوشته شده روی هر راس $v$ را با $n(v)$ نشان می‌دهیم. همچنین عمق هر راس $v$ را با $d(v)$ نشان می‌دهیم که نشان دهنده‌ی فاصله‌ی $v$ تا ریشه اشت.
	\item در هر مرحله یک راس که قبلا گسترش پیدا نکرده است را گسترش می‌دهیم. اگر در مرحله‌ی تمام راس‌ها گسترش پیدا کرده بودند، الگوریتم متوقف می‌شود.
	\item برای هر راس مانند $v$ اگر مسیر ریشه تا این راس را با
	\lr{$a_1$, $a_2$, ... ,$a_{(d(v)}$}
	نشان دهیم، توافق می‌کنیم که راس $v$ متناظر با دنباله‌ی
	$n(a_2), ..., n(a_{d(v)})$
	است. خود ریشه متناظر با دنباله‌ی درنظر گرفته می‌شود.
\end{itemize}
باید معنی "گسترش" در الگوریتم بالا را تعریف کنیم.
 منظور از گسترش $v$ این است که با حالت بندی اینکه مقدار $v_{i+1}$ چند حالت برای گرد شدن دارد (اگر صحیح بود دقیقا ۱ حالت وگرنه دقیقا ۲ حالت)، برای هر حالت اگر مقدار گرد شده را با $z_{i+1}$ نشان دهیم، یک بچه برای $v$ بسازیم و روی آن مقدار $z_{i+1}$ را بنویسیم.  البته برای هر کدام از $z_{i+1}$ های ممکن، فقط در صورتی این بچه را می‌سازیم که دنباله‌ی متناظر با این بچه (که در الگوریتم بالا این دنباله را تعریف کرده‌ایم)، تمام شروط بازه‌ای به طوری که $j \leq i+1$ را ارضا کند. 

تا اینجا درخت $T$ را تعریف کردیم. حال ادعا می‌کنیم که $T$ نامتناهی است. فرض کنید این طور نباشد (برهان خلف)، بنابراین ساخت $T$ در یک مرحله متوقف می‌شود و یک عمق $k$ وجود دارد که برابر با بیشینه‌ی عمق رئوس $v \in V(T)$ است. از طرفی طبق قسمت الف، می‌دانیم که با قرار دادن $n= d(v)+1$، می‌توان یک دنباله‌ی $n$ عضوی ساخت که تمام قیود بازه‌ای با فرض اینکه $j \leq n$ را برقرار می‌کند، بنابراین الگوریتم نمی‌تواند در عمق $k$ متوقف می‌شود و حداقل تا عمق $k+1$ ادامه پیدا می‌کند که با فرض خلف در تناقض است. در نتیجه درخت $T$ نامتناهی است.

حال توجه کنید که نحوه‌ی تعریف $T$ به گونه‌ای است که درجه‌ی هر راس حداکثر ۳ است (حداکثر ۲ بچه و حداکثر ۱ پدر). بنابراین از لم نامتناهی  کونیگ نتیجه می‌شود که یک مسیر نامتناهی داریم. همچنین می‌توانیم فرض کنیم که یک سر این مسیر برابر با ریشه‌ی درخت است (در غیر اینصورت می‌توان مسیر نامتناهی $P$ را درنظر گرفت و یکی از دو طرف آن را به اجبار به ریشه‌ی درخت تغییر داد به طوری که $P$ همچنان نامتناهی بماند). از طرفی این مسیر با توچه به اینکه از ریشه شروع می‌شود، دقیقا متناظر با یک دنباله‌ی نامتناهی از $z_i$ ها است که $z_i$ ها دقیقا همان اعداد نوشته‌ شده بر روی راس‌های این مسیر است که همان حکم سوال است.
\begin{latin}
	$\square$
\end{latin}
\end{document}
