\documentclass[a4paper,12pt]{article}
\usepackage{HomeWorkTemplate}
\usepackage{circuitikz}
\usepackage[shortlabels]{enumitem}
\usepackage{hyperref}
\usepackage{tikz}
\usepackage{listings}

\def\Max{\text{بیشینه کن}}
\def\Min{\text{کمینه کن}}
\def\st{\text{\rl{که}}}

\usetikzlibrary{shapes.multipart}

\usepackage{amsmath}
\usepackage{amssymb}
\usepackage{tcolorbox}
\usepackage{xepersian}
\settextfont{XB Niloofar}
\usetikzlibrary{arrows,automata}
\usetikzlibrary{circuits.logic.US}
\usepackage{changepage}
\newcounter{problemcounter}
\newcounter{subproblemcounter}
\setcounter{problemcounter}{1}
\setcounter{subproblemcounter}{1}
\newcommand{\problem}[1]
{
	\subsection*{
		پرسش
%		\arabic{problemcounter} 
%		\stepcounter{problemcounter}
%		\setcounter{subproblemcounter}{1}
		#1
	}
}
\newcommand{\subproblem}{
	\textbf{\harfi{subproblemcounter})}\stepcounter{subproblemcounter}
}


\begin{document}
\handout
{بهینه‌سازی ترکیبیاتی مقدماتی}
{مرتضی علیمی، هانی احمد زاده}
{بهار 1400}
{اطلاعیه}
{سروش زارع}
{97100405}
 {تمرین تحویلی 8}
\paragraph{\color{blue}
در سوال ۲ ماتروید‌ها را به صورت
$M = (S, \mathcal{I})$
و
$M^* = (S, \mathcal{I'})$
فرض می‌کنیم. همچنین منظور از عبارتی مانند
$span_M(x)$،
$span(x)$
در ماتروید $M$ و منظور از عبارتی مانند
$span_{M'}(x)$،
$span(x)$
در ماتروید $M'$ می‌باشد(به جای
$span$
هر تابع دیگری ممکن است استفاده شود).
}
\problem{2}
از برهان خلف استفاده می‌کنیم، فرض کنید حکم برقرار نباشد و درنتیجه مدار‌های $C$ و $C^*$ فقط در یک $x$ اشتراک داشته باشند.
طبق تعریف دوگان، می‌دانیم که 
$A \in \mathcal{I'}$
 اگر و تنها اگر پایه‌ای برای $M$ در مجموعه‌ی 
 $S \backslash A$
  وجود داشته باشد. بنابراین طبق تعریف زنجیر می‌دانیم:
\begin{align}
	Y \in \mathcal{I'} \quad \forall Y \subset C^* \rightarrow C^* - x \in \mathcal{I'}
\end{align}
\lemma{ادعا می‌کنیم که
\begin{align*}
 x \notin span_M(S \backslash C^*)	
\end{align*}
}
\proof{}
این ادعا با برهان خلف اثبات می‌شود، فرض کنید این طور نباشد و در نتیجه 
$x \in span_M(S \backslash C^*)	$
بنابراین  داریم
\begin{align}
	rank(span_M(S \backslash C^*)) = rank(span_M(S \backslash C^* + x))
\end{align}
بنابراین در عبارت (۱)، می‌توانیم $x$ را نیز به مجموعه‌ی 
$Y = C^* - x$
اضافه کنیم و $Y$ حاصل همچنان عضو $\mathcal{I'}$ باقی می‌ماند (زیرا طبق (۲)، 
$rank_M(S - Y)$
کاهش نمی‌یابد). از طرفی می‌دانستیم که $C^*$ یک مدار است و عضو $\mathcal{I'}$ نیست، از تناقض حاصل نتیجه می‌گیریم که فرض خلف باطل است و لم برقرار است.
\begin{latin}
	$\square$
\end{latin}

\lemma{
مجموعه‌ی
$S \backslash C^*$
را درنظر بگیرید. از آنجایی که $C$ یک مدار است، $C - x$ یک عضو مستقل برای $M$ است و می‌توانیم آن را به یک پایه برای 
$S \backslash C^*$
 گسترش دهیم. پایه‌ی حاصل را $B$ بنامید. ادعا می‌کنیم که 
$B' := B + x$
 همچنان مستقل است.
}
\proof{}
فرض کنید اینطور نباشد (برهان خلف). بنابراین داریم:
\begin{align*}
	x \in span_M(S \backslash C^* )
\end{align*}
که به طور مستقیم با لم (۱) در تناقض است. بنابراین فرض خلف باطل است و لم(۲) اثبات می‌شود.

\lemma{
 $B'$ به دست آمده در لم (۲) را درنظر بگیرید   
داریم:
\begin{align*}
	C \subseteq B' \rightarrow C \in \mathcal{I}
\end{align*}
}
\proof{}
طبق اصول موضوعه‌ی ماتروید، داریم:
\begin{align*}
	A \in \mathcal{I} , B \subseteq A \rightarrow B \in \mathcal{I}
\end{align*}
که دقیقا مطابق با لم ۳ است (توجه کنید دلیل اینکه
$C \subseteq B'$
این است که هنگام ساختن $B$ ابتدا از عضو مستقل 
$C - x$
شروع کردیم و در نهایت $x$ را به $B$ اضافه کردیم، بنابراین تمام اعضای $C$ در $B'$ قرار دارند و $C \subseteq B'$).

لم ۳ به وضوح با اینکه $C$ یک زنجیر است و درنتیجه 
$C \notin \mathcal{I}$
 در تناقض است. پس فرض خلف باطل است و امکان ندارد که
$|C \cap C^*| = 1$.
در نتیجه حکم سوال اثبات شد. 
\end{document}
