\documentclass[a4paper,12pt]{article}
\usepackage{HomeWorkTemplate}
\usepackage{circuitikz}
\usepackage[shortlabels]{enumitem}
\usepackage{hyperref}
\usepackage{tikz}
\usepackage{listings}

\def\Max{\text{بیشینه کن}}
\def\Min{\text{کمینه کن}}
\def\st{\text{\rl{که}}}

\usetikzlibrary{shapes.multipart}

\usepackage{amsmath}
\usepackage{amssymb}
\usepackage{tcolorbox}
\usepackage{xepersian}
\settextfont{XB Niloofar}
\usetikzlibrary{arrows,automata}
\usetikzlibrary{circuits.logic.US}
\usepackage{changepage}
\newcounter{problemcounter}
\newcounter{subproblemcounter}
\setcounter{problemcounter}{1}
\setcounter{subproblemcounter}{1}
\newcommand{\problem}[1]
{
	\subsection*{
		پرسش
%		\arabic{problemcounter} 
%		\stepcounter{problemcounter}
%		\setcounter{subproblemcounter}{1}
		#1
	}
}
\newcommand{\subproblem}{
	\textbf{\harfi{subproblemcounter})}\stepcounter{subproblemcounter}
}


\begin{document}
\handout
{بهینه‌سازی ترکیبیاتی مقدماتی}
{مرتضی علیمی، هانی احمد زاده}
{بهار 1400}
{اطلاعیه}
{سروش زارع}
{97100405}
 {تمرین تحویلی 8}
\problem{1}
ابتدا چند لم ابتدایی مربوط به گراف‌ها را اثبات می‌کنیم و سپس به سراغ اثبات حکم مساله می‌رویم.

\paragraph{
یک گراف همبند $C$ را درنظر بگیرید، اگر تعداد رئوس و یال‌های $C$ را به ترتیب با $V(C)$ و $E(C)$ نشان دهیم، 2 لم زیر برقرارند:
}



\lemma{
در صورتی که
$E(C) \geq V(C)$
در $C$ یک دور وجود دارد.
به طور دقیق تر، اگر این نامساوی به صورت تساوی برقرار باشد دقیقا یک دور در $C$ وجود دارد.
}

\proof{
از آنجایی که $C$ همبند است می‌توان یک زیردرخت فراگیر $T$ با $V(C) - 1$ یال برای آن تشکیل داد.  از طرفی داریم:
\begin{align*}
	E(C) \geq V(C) \rightarrow E(C) > V(C) - 1
\end{align*}
در نتیجه حداقل یک یال $e \in E(C)$ وجود دارد که 
$e \notin E(T)$.
اضافه کردن این یال به $T$ به وضوح یک دور ایجاد می‌کند. همچنین اگر فقط یک انتخاب برای $e$ داشته باشیم، دقیقا یک دور ایجاد می‌شود. درنتیجه لم اثبات می‌شود.
}

\lemma{
در صورتی که
$E(C) > V(C)$
در $C$ اکیدا بیش از ۱ دور وجود دارد.
}

\proof{
از برهان خلف استفاده کنید و فرض کنید اینطور نباشد، پس طبق لم قبلی، دقیقا یک دور در $C$ وجود دارد. یک یال دلخواه $e$ از این دور را درنظر بگیرید و حذف کنید و گراف حاصل را $C'$ بنامید. اگر $C'$ همچنان همبند بماند، داریم:
\begin{align}
	E(C') = E(C - 1) \rightarrow E(C') \geq V(C') = V(C)
\end{align}
و طبق لم ۱، همچنان یک دور در $C'$ وجود دارد که با دور قبلی در مجموع ۲ دور می‌شود و لم اثبات می‌شود. پس فرض کنید که $C'$ همبند نباشد. اگر مولفه‌های همبندی $C'$ را با
$C_1, ... C_k$
نشان دهیم، اگر حداقل یک مولفه‌ی
$C_i$
به اندازه‌ی 
$V(C_i)$
یال داشته باشد، درون این مولفه یک دور وجود دارد که با دور اولیه می‌شود ۲ دور و لم اثبات می‌شود. پس فرض می‌کنیم:
\begin{align*}
	E(C_i) < V(C_i) \quad \forall 1 \leq i \leq k
\end{align*}
و از آنجایی که یالی بین این مولفه‌های همبندی وجود ندارد، داریم:
\begin{align*}
	E(C') = \sum_i^k E(C_i) \leq \sum_i^k (V(C_i) - 1) \leq (\sum_i^k V(C_i)) - k = V(C') - k
\end{align*}
با توجه به اینکه فرض کردیم $C'$ همبند نیست، داریم:
\begin{align*}
	k \geq 2 \rightarrow E(C') \leq V(C') - 2
\end{align*}
که با نامساوی (۱) در تناقض است. بنابراین فرض خلف باطل است و لم (۲) نیز برقرار است.
}

\paragraph{حال ۲ لم بالا را به عنوان فرض قبول می‌کنیم و به سراغ مساله‌ی اصلی می‌رویم:}

\lemma{
اگر 
$A \in \mathcal{I}$
به ازای هر 
$B \subseteq A$
نیز داریم
$B \in \mathcal{I}$.
}
\proof{}
به وضوح اگر گردایه‌ای از یال‌ها شامل حداکثر ۱ دور باشد ($A$)، هر زیرمجموعه‌ای از این یال‌ها ($B$) نیز حداکثر ۱ دور دارد، بنابراین لم اثبات می‌شود.

\lemma{
اگر 
$A, B \in \mathcal{I}$
و 
$|A| < |B|$
می‌توان یک 
$x \in B \backslash A$
پیدا کرد به طوری که
$A + x \in \mathcal{I}$
}

\proof{}
مولفه‌های همبندی متناظر با یال‌های $A$ و رئوس $V$ را با
$C_1, ... C_k$
نشان دهید. به وضوح هیچ یالی بین دو $C_i$ متفاوت وجود ندارد (وگرنه روی هم یک مولفه‌ی همبندی بودند). با توجه به لم‌های (۱) و (۲)، حداکثر در یکی از $C_i$ ها ممکن است داشته باشیم
$E(C_i) = V(C_i)$
و به ازای تک تک $C_i$ های دیگر داریم
$E(C_i) < V(C_i)$
(وگرنه $E$ حداقل ۲ دور تشکیل می‌دهد). از طرفی با توجه به اینکه هر $C_i$ یک مولفه‌‌ی همبندی است، داریم:
$E(C_i) \geq V(C_i) - 1$.
بنابراین به ازای حداقل 
$k-1$
مولفه‌ی 
$C_i$
داریم:
\begin{align*}
E(C_i) = V(C_i) - 1
\end{align*}
و به ازای حداکثر یک مولفه‌ی $C_i$ داریم:
\begin{align*}
E(C_i) = V(C_i)	
\end{align*}
بنابراین اگر تعداد کل‌یال‌های $A$ (یا همان $|A|$) را حساب کنیم، داریم:
\begin{align}
	|A| \leq (\sum_i^k (V(C_i) - 1)) + 1 = |V| - k + 1 \\
	|A| \geq (\sum_i^k (V(C_i) - 1)) = |V| - k
\end{align}

حال یال‌های $B$ را درنظر بگیرید و آن‌ها را با رنگ قرمز در نظر بگیرید.
مجددا طبق لم‌های ۱ و ۲،  یال‌های قرمز درون هیچ $C_i$ نمی‌تواند اکیدا از $V(C_i)$ بیشتر باشد وگرنه $B$ حداقل ۲ دور دارد. پس اگر یال‌های قرمز درون مولفه‌ی 
$C_i$
را با
$R(C_i)$
نشان دهیم، داریم:
\begin{align*}
	R(C_i) \leq V(C_i) \quad \forall 1 \leq i \leq k
\end{align*}
با همان استدلال مشابه برای $A$، می‌توان دید که این نامساوی نیز حداکثر در یک $i$ به صورت تساوی برقرار است.
حال از برهان خلف استفاده می‌کنیم و فرض می‌کنیم که هیچ 
$x \in B \backslash A$
وجود نداشته باشد که 
$A + x \in \mathcal{I}$.
بنابراین هیچ یال قرمزی بین دو مولفه‌ی متفاوت 
$C_i$
و
$C_j$
وجود ندارد (وگرنه $x$ را می‌توانستیم همان یال درنظر بگیریم و تعداد دور‌های 
$A + x$
نسبت به
$A$
افزایش پیدا نمی‌کرد). بنابراین تعداد کل‌ یال‌های قرمز  یا همان $|B|$ حداکثر برابر است با:
\begin{align*}
 	|B| \leq (\sum_i^k (V(C_i) - 1)) + 1 = |V| -k + 1
\end{align*}

از  آنجایی که  فرض کردیم $|B| > |A|$، بنابراین تنها حالت معتبر این است که 
$|A| = |V| - k$
باشد و درنتیجه $A$ هیچ دوری تشکیل ندهد و 
$|B| = |V| - k + 1$
باشد و درنتیجه $B$ درون یکی از مولفه‌های $C_i$ تشکیل دور می‌دهد. به راحتی می‌توانیم یک یال $e \notin A$ درون این دور را انتخاب کنیم و آن را به $A$ اضافه کنیم (اگر دور را با $D$ نشان دهیم، حداقل یک یال $D$وجود دارد که در $A$ وجود ندارد، زیرا در غیر اینصورت $A$ نیز شامل دور $D$ می‌شود که با فرض دور نداشتن $A$ در تناقض است، پس $e \notin A$ وجود دارد) از آنجایی که  $A$ قبلا در $C_i$ دور نداشت، اکنون دقیقا ۱ دور در $C_i$ دارد و در $C_i$ های دیگر نیز دور ندارد، بنابراین
$A + e$
دقیقا یک دور دارد و
$A + e \in \mathcal{I}$
و درنتیجه لم اثبات می‌شود.

\paragraph{
با توجه به لم‌های اثبات‌شده‌ی (۳) و (۴) که اصول موضوعه‌ی مربوط به ماتروید‌ها هستند، 
$(E,\mathcal{I})$
 یک ماتروید است و حکم مساله اثبات می‌شود.\\}
\end{document}
