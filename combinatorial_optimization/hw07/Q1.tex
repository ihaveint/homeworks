\documentclass[a4paper,12pt]{article}
\usepackage{HomeWorkTemplate}
\usepackage{circuitikz}
\usepackage[shortlabels]{enumitem}
\usepackage{hyperref}
\usepackage{tikz}
\usepackage{listings}

\def\Max{\text{بیشینه کن}}
\def\Min{\text{کمینه کن}}
\def\st{\text{\rl{که}}}

\usetikzlibrary{shapes.multipart}

\usepackage{amsmath}
\usepackage{amssymb}
\usepackage{tcolorbox}
\usepackage{xepersian}
\settextfont{XB Niloofar}
\usetikzlibrary{arrows,automata}
\usetikzlibrary{circuits.logic.US}
\usepackage{changepage}
\newcounter{problemcounter}
\newcounter{subproblemcounter}
\setcounter{problemcounter}{1}
\setcounter{subproblemcounter}{1}
\newcommand{\problem}[1]
{
	\subsection*{
		پرسش
%		\arabic{problemcounter} 
%		\stepcounter{problemcounter}
%		\setcounter{subproblemcounter}{1}
		#1
	}
}
\newcommand{\subproblem}{
	\textbf{\harfi{subproblemcounter})}\stepcounter{subproblemcounter}
}


\begin{document}
\handout
{بهینه‌سازی ترکیبیاتی مقدماتی}
{مرتضی علیمی، هانی احمد زاده}
{بهار 1400}
{اطلاعیه}
{سروش زارع}
{97100405}
 {تمرین تحویلی 7}
\problem{1}
الگوریتم زیر را درنظر بگیرید:
\begin{enumerate}
	\item قرار بده $C = \{y\}$.
	\item اگر عضوی از $A \backslash C$ مانند $t$ وجود داشت به طوری که $C + t \in \mathcal{I}$، آنگاه قرار بده $C = C + t$ و باز به همین مرحله‌ی ۲ برگرد. درغیر اینصورت الگوریتم به پایان می‌رسد و خروجی را برابر با $C$ قرار بده.
\end{enumerate}

با توجه به تعریف این الگوریتم، واضح است که خروجی $C$ عضو $\mathcal{I}$ است (زیرا در هر مرحله $C$ طوری بزرگ می‌شود که این شرط برقرا بماند).
\lemma{اگر خروجی الگوریتم را با $C$ نشان دهیم، $C$ دقیقا برابر است با $A - x +y$}
\proof{}
فرض کنید این طور نباشد(برهان خلف)، از آنجایی که در هربار اجرای الگوریتم $|C|$ دقیقا یک واحد افزایش پیدا می‌کند و با توجه به اینکه $x$ و $y$ دو عنصر موازی هستند، هیچ گاه $x$ و $y$ هر دو در $C$ قرار نمی‌گیرند (در غیر اینصورت در یک مرحله خواهیم داشت
\begin{align*}
	x , y \in C, C \in \mathcal{I} \rightarrow \{x,y\} \in \mathcal{I}
\end{align*} 
دلیل این مورد این است که طبق اصول موضوعه‌ی ماتروید،  داریم:
\begin{align*}
\{x,y\} \subseteq C , C \in \mathcal{I} \rightarrow \{x,y\} \in \mathcal{I}
\end{align*}
که با فرض اینکه $x$ و $y$ دو عنصر موازی‌اند در تناقض است).
پس از آنجایی که $C$ در ابتدا شامل $y$ است، نتیجه می‌گیریم هیچ گاه $x$ درون $C$ قرار نمی‌گیرد. بنابراین در هر بار اجرای الگوریتم، یک عنصر از $A - x$ به $C$ اضافه می‌شود و در نهایت پس از تعدادی بار این روند متوقف می‌شود، اگر تمام اعضای $A - x$ به $C$ اضافه نشوند(همان فرض خلف)، در نهایت داریم:
\begin{align*}
|C| < |A - x + y| = |A|	
\end{align*}
از طرفی طبق تعریف ماتروید، می‌دانیم که با توجه به اینکه
$|A| > |C| , A,C \in \mathcal{I}$
، یک عضو
$t \in A \backslash C$
پیدا می‌شود که
$t + C \in \mathcal{I}$.
با کمی دقت می‌بینیم که مرحله‌ی ۲ الگوریتم دقیقا در تلاش برای پیدا کردن همین $t$ بود، بنابراین الگوریتم نمی‌تواند در این مرحله به پایان رسیده باشد و از تناقض حاصل نتیجه می‌گیریم که فرض خلف باطل است و $C$نهایی دقیقا برابر است با
$A - x +y$.
با توجه به اینکه در هر بار اجرای الگوریتم شرط
$C \in \mathcal{I}$
برقرار می‌ماند، در نهایت داریم:
\begin{align*}
	A - x + y = C \in \mathcal{I}
\end{align*}
که همان حکم سوال است.
\end{document}
