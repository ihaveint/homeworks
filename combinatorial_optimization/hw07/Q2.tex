\documentclass[a4paper,12pt]{article}
\usepackage{HomeWorkTemplate}
\usepackage{circuitikz}
\usepackage[shortlabels]{enumitem}
\usepackage{hyperref}
\usepackage{tikz}
\usepackage{listings}

\def\Max{\text{بیشینه کن}}
\def\Min{\text{کمینه کن}}
\def\st{\text{\rl{که}}}

\usetikzlibrary{shapes.multipart}

\usepackage{amsmath}
\usepackage{amssymb}
\usepackage{tcolorbox}
\usepackage{xepersian}
\settextfont{XB Niloofar}
\usetikzlibrary{arrows,automata}
\usetikzlibrary{circuits.logic.US}
\usepackage{changepage}
\newcounter{problemcounter}
\newcounter{subproblemcounter}
\setcounter{problemcounter}{1}
\setcounter{subproblemcounter}{1}
\newcommand{\problem}[1]
{
	\subsection*{
		پرسش
%		\arabic{problemcounter} 
%		\stepcounter{problemcounter}
%		\setcounter{subproblemcounter}{1}
		#1
	}
}
\newcommand{\subproblem}{
	\textbf{\harfi{subproblemcounter})}\stepcounter{subproblemcounter}
}


\begin{document}
\handout
{بهینه‌سازی ترکیبیاتی مقدماتی}
{مرتضی علیمی، هانی احمد زاده}
{بهار 1400}
{اطلاعیه}
{سروش زارع}
{97100405}
 {تمرین تحویلی 7}
\problem{2}
اندیس‌هایی از $s_i$ که در $T$ قرار دارند را با
$c_1, ... c_k$
نشان دهید. بنابراین داریم:
\begin{align*}
	T = \{s_{c_1}, ... s_{c_k}\}
\end{align*}
همچنین تعریف می‌کنیم:
\begin{align*}
	V_i &= \{s_1, ... s_{c_i - 1} \} \quad \forall 1 \leq i \leq k \\
	U_i &= \{s_{c_1}, ..., s_{c_{i - 1}}\} \quad \forall 1 \leq i \leq k \\
	V_{k+1} &= \{s_1, ..., s_m\} \\
	U_{k + 1} &= \{s_{c_1}, ... s_{c_k}\} = T
\end{align*}
توجه کنید که طبق این تعاریف داریم:
\begin{align*}
	V_{i + 1} = V_{i} + s_{c_i} \\
	rank(V_1) = 0
\end{align*}
طبق تعریف $T$، داریم:
\begin{align}
	rank(V_i) < rank(V_i + s_{c_i}) \quad \forall 1 \leq i \leq k
\end{align}

\lemma{
داریم:
\begin{align*}
	rank(V_i + s_{c_i}) = rank(V_i) + 1 \quad \forall 1 \leq i \leq k
\end{align*}
}

\proof{
با توجه به نامساوی (۱)، کافیست اثبات کنیم
\begin{align*}
	rank(V_i + s_{c_i}) \leq rank(V_i) + 1 \quad \forall 1 \leq i \leq k
\end{align*}
این موضوع نیز تقریبا واضح است، فرض کنید این طور نباشد و یک مجموعه‌ی مستقل 
$b_1, ..., b_l$
برای 
$V_i + s_{c_i}$
وجود داشته باشد به طوری که
$l > rank(V_i) + 1$.
از آنجایی که حداکثر یکی از این $b_i$ ها برابر با $s_{c_i}$ می‌باشد، با حذف آن $b_i$ یک مجموعه‌ی مستقل $B$ به دست می‌آید به طوری که: 
\begin{align*}
	|B| \geq l - 1 > rank(V_i)
\end{align*}
از طرفی با توجه به اینکه 
$|B| \subseteq V_i + s_{c_i} - s_{c_i} = V_i$
خود $|B|$ یک مجموعه‌ی مستقل برای $V_i$ است و این با فرض اینکه 
$rank(V_i) < l -1$
در تناقض است. از تناقض حاصل نتیجه می‌گیریم که
\begin{align*}
	rank(V_i + s_{c_i}) \leq rank(V_i) + 1 \quad \forall 1 \leq i \leq k
\end{align*}
و از ترکیب این نامساوی و نامساوی (۱) حکم لم اثبات می‌شود.
}


\lemma{
داریم:
\begin{align*}
	rank(U_i) = rank(V_i) \quad \forall 1 \leq i \leq k+1
\end{align*}
}
\proof{}
با توجه به اینکه
$U_i \subseteq V_i$،
بنابراین 
\begin{align}
	rank(U_i) \leq rank(V_i) \quad \forall 1 \leq i \leq k+1
\end{align}
از طرفی فرض کنید یک 
$1 \leq i \leq k+1$
وجود داشته باشد که
$rank(V_i) > rank(U_i)$.
در این صورت در هنگام ساخته شدن $T$، حداقل یک عضو 
$V_i \backslash U_i$
نیز باید در $T$ قرار می‌گرفت و $T$ به اشتباه ساخته شده است، که تناقض حاصل نشان می‌دهد 
\begin{align}
	rank(V_i) \leq rank(U_i) \quad \forall 1 \leq i \leq k+1
\end{align}
از ۲ نامساوی (2) و (3) نتیجه می‌گیریم:
\begin{align}
	rank(U_i) = rank(V_i) \quad \forall 1 \leq i \leq k + 1
\end{align}
که همان حکم لم است.

\lemma{
داریم:
\begin{align*}
	rank(U_i) = |U_i| = i - 1 \quad \forall 1 \leq i \leq k+1
\end{align*}
و درنتیجه تمام $U_i$ ها مستقل هستند.
}
\proof{}
از لم (۱) می‌توانیم نتیجه بگیریم که
\begin{align*}
	rank(V_i) = i - 1 \quad \forall 1 \leq i \leq k+1
\end{align*}
که از ترکیب این تساوی با لم (۲) داریم:
\begin{align*}
	rank(U_i) = rank(V_i) = i - 1 \quad \forall 1 \leq i \leq k + 1
\end{align*}
با توجه به اینکه هر $U_i$ دقیقا $i - 1$ عضو دارد، داریم:
\begin{align*}
	rank(U_i) = i - 1 = |U_i| \quad \forall 1 \leq i \leq k + 1
\end{align*}
پس لم(۳) اثبات می‌شود.
\newline
\newline
در نتیجه‌ی  لم(۳)، 
$U_{k+1} = T$
نیز مستقل است که همان حکم سوال است.
\begin{latin}
	$\square$
\end{latin}
\end{document}
