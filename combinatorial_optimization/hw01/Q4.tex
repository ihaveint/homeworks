\documentclass[a4paper,12pt]{article}
\usepackage{HomeWorkTemplate}
\usepackage{circuitikz}
\usepackage[shortlabels]{enumitem}
\usepackage{hyperref}
\usepackage{tikz}
\usepackage{listings}

\usetikzlibrary{shapes.multipart}

\usepackage{amsmath}
\usepackage{amssymb}
\usepackage{tcolorbox}
\usepackage{xepersian}
\settextfont{XB Niloofar}
\usetikzlibrary{arrows,automata}
\usetikzlibrary{circuits.logic.US}
\usepackage{changepage}
\newcounter{problemcounter}
\newcounter{subproblemcounter}
\setcounter{problemcounter}{1}
\setcounter{subproblemcounter}{1}
\newcommand{\problem}[1]
{
	\subsection*{
		پرسش
%		\arabic{problemcounter} 
%		\stepcounter{problemcounter}
%		\setcounter{subproblemcounter}{1}
		#1
	}
}
\newcommand{\subproblem}{
	\textbf{\harfi{subproblemcounter})}\stepcounter{subproblemcounter}
}


\begin{document}
\handout
{بهینه‌سازی ترکیبیاتی مقدماتی}
{مرتضی علیمی، هانی احمد زاده}
{نیم‌سال اول 1399\lr{-}1400}
{اطلاعیه}
{سروش زارع}
{97100405}
 {تمرین تحویلی 1}
\problem{4}
اثبات این سوال در جزوه آمده است که آن را در اینجا می‌آوریم.
\newline
چندوجهی جایگشت‌ها را با
 با $\pi_n$
 نشان می‌دهیم.
\newline
ابتدا یک راس
$x \in \pi_n$
را در نظر بگیرید. این راس متناظر با یک جایگشت است و  و به وضوح در مقدار ادعا شده برای $conv(X)$  قرار دارد زیرا جمع عناصر آن دقیقا
$n(n+1)/2$
است و هر $i$ عنصری از آن جمع حداقل
$i(i+1)/2$
دارند (با توجه به متمایز بودن درایه‌ها و طبیعی بودن آن‌ها و همچنین اینکه حداقل مقدار ممکن برای درایه‌ها برابر با ۱ است).
\newline
\newline
حال یک راس $y$ درون $conv(X)$ ادعا شده درنظر بگیرید. ثابت می‌کنیم که
$y$
درون چندوجهی جایگشت‌ها (که آن را با $\pi_n$) قرار دارد و با این تفسیر اگر تمام راس‌های $conv(X)$ ادعا شده درون چندوجهی $\pi_n$ قرار داشته باشد، تمام نقاط $conv(X)$ درون این چندوجهی قرار دارد و در نتیجه‌ی اینکه هر کدام از 
$pi_n$
و $conv(X)$ زیر مجموعه‌ی دیگری هستند، 
$conv(X)$
ادعا شده دقیقا همان
چندوجهی جایگشت‌ها یا
$\pi_n$
است که همان حکم سوال است.

پس کافیست سمت باقی مانده‌ی ادعا را اثبات کنیم:
یک راس دلخواه
$y$
درون $conv(X)$ ادعا شده درنظر بگیرید. با توجه به اینکه $y$ یک راس است، یک بردار $c$ وجود دارد که $y$ جواب یکتای
$max\{c^Tx | x \in conv(x) \}$
است.  اگر درایه‌های 
$c_i$
را به ترتیب
$c_{i_1} \leq c_{i_2} \leq ... c_{i_n}$
در نظر بگیریم، ادعا می‌کنیم که برای هر 
$1 \leq k \leq n$
داریم
$y_{i_k} = k$
و با اثبات این ادعا مشخص می‌شود که $y$ متناظر با یک جایگشت خواهد بود.
اثبات این تساوی با برهان خلف و درنظر گرفتن کوچک ترین اندیس $k$ که در این تساوی صدق نمی‌کند به دست می‌آید که در صفحه‌ی ۴۸ جزوه و در مثال جایوجهی آمده است که در اینجا آن را تکرار نمی‌کنیم.
\end{document}