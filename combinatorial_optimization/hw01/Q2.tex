\documentclass[a4paper,12pt]{article}
\usepackage{HomeWorkTemplate}
\usepackage{circuitikz}
\usepackage[shortlabels]{enumitem}
\usepackage{hyperref}
\usepackage{tikz}
\usepackage{listings}

\usetikzlibrary{shapes.multipart}

\usepackage{amsmath}
\usepackage{amssymb}
\usepackage{tcolorbox}
\usepackage{xepersian}
\settextfont{XB Niloofar}
\usetikzlibrary{arrows,automata}
\usetikzlibrary{circuits.logic.US}
\usepackage{changepage}
\newcounter{problemcounter}
\newcounter{subproblemcounter}
\setcounter{problemcounter}{1}
\setcounter{subproblemcounter}{1}
\newcommand{\problem}[1]
{
	\subsection*{
		پرسش
%		\arabic{problemcounter} 
%		\stepcounter{problemcounter}
%		\setcounter{subproblemcounter}{1}
		#1
	}
}
\newcommand{\subproblem}{
	\textbf{\harfi{subproblemcounter})}\stepcounter{subproblemcounter}
}


\begin{document}
\handout
{بهینه‌سازی ترکیبیاتی مقدماتی}
{مرتضی علیمی، هانی احمد زاده}
{نیم‌سال اول 1399\lr{-}1400}
{اطلاعیه}
{سروش زارع}
{97100405}
 {تمرین تحویلی 1}
\problem{2}
\subproblem
قضیه رادون!
(صورت قضیه برای یادآوری در زیر آورده شده است)
\theorem{فرض کنید $S \in \mathbb{R}^d$. اگر $d+2$ نقطه‌ی دلخواه داخل $S$ بگیریم و این نقاط را $A$ بنامیم، می‌توان $A$ را به دو دسته افراز کرد که پوش محدب این دو دسته اشتراک داشته باشد. \\}
در واقع قضیه‌ی رادون، حالت خاص قسمت (الف) برای $r=2$ است.
\newline
\newline
\subproblem
حالت (الف) را به حالت مخروطی در (ب) کاهش می‌دهیم و بنابراین درستی حالت الف اثبات می‌شود.
مجموعه‌ی $A$ را همان مجموعه‌ی داخل مساله‌ی (الف) درنظر بگیرید. با توجه به اینکه این نقاط در $\mathbb{R}^d$ قرار دارند، می‌توانیم یک بعد دیگر اضافه کنیم و قرار دهیم:
\begin{align*}
	B_i := (x_i, 1) \forall x_i \in A
\end{align*}
در واقع یک اسکالر ۱ را به عنوان آخرین عضو به $x_i$ اضافه می‌کنیم تا $B_i$ حاصل شود.
\lemma{\lr{CH(B)} شامل مبدا نیست.}
\proof{}
اثبات به راحتی با تعریف ترکیب محدب حاصل می‌شود، داریم:
\begin{align*}
	CH(B) = \{x | x = \sum_{x_i \in B}\lambda_i x_i , \sum_i \lambda_i = 1, 0 \leq \lambda_i \leq 1  \}
\end{align*}
حال اگر توجهمان را متمرکز به آخرین اندیس $x_i$ ها  کنیم، می‌بینیم که هر ترکیب محدب از عنصر‌های $B$، آخرین اندیسشان دقیقا ۱ خواهد بود. به طور دقیق تر داریم:
\begin{align*}
	y \in CH(B) &\rightarrow y = \sum_{x_i \in B}\lambda_i x_i , \sum_i \lambda_i = 1, 0 \leq \lambda_i \leq 1 \\
	&\rightarrow y_{d + 1} = \sum_{x_i \in B}\lambda_i {(x_i)}_{d + 1} = \sum_{x_i \in B}\lambda_i = 1 \\
	&\rightarrow y \neq 0 
\end{align*}
که همان شرط لازم برای برقراری مساله‌ی (ب) به ازای بردار‌های 
$x \in B$
است.
بنابراین چون قسمت (ب) را به عنوان فرض قبول کردیم، می‌توانیم $B$ را به زیرمجموعه‌های
$B_1, ... B_r$
تقسیم کنیم به طوری که
$\bigcap_i^r cone(B_i) \neq \{0\}$.
یکی از اعضای این اشتراک را با $D$ نشان دهید.  با توجه به نحوه‌ی تعریف $B$ و همچنین تعریف تابع cone ، $(d+1)$امین اندیس $D$ اکیدا بیشتر از ۰ است. همچنین طبق تعریف cone می‌توانیم ببینیم که هر ضریب 
$\lambda D : \lambda \geq 0$
 نیز عضو این اشتراک خواهد بود. یکی از این ضریب‌ها مانند $D'$ را درنظر می‌گیریم که $(d+1)$امین اندیس‌ آن دقیقا برابر با ۱ باشد (با قرار دادن 
$\lambda = \frac{1}{D_{d + 1}}$
این کار ممکن است). بنابراین داریم:
\begin{align*}
	&\forall 1 \leq i \leq r \  \exists \lambda_1 \geq 0, ... ,\lambda_{|B_i|} \geq 0 : D' = \sum_j^{|B_i|} \lambda_j {(B_i)}_{j} , D'_{d + 1} = 1 \\ 
	&\rightarrow \sum_j^{|B_i|}\lambda_j = 1 \rightarrow D' \in \bigcap_i^r CH(B_i)
\end{align*}
که منظور از 
${(B_i)}_{j}$
همان
$j$امین عنصر از مجموعه‌ی 
$B_i$
است و منظور از $D'_{d+1}$ درایه‌ی $d+1$ از $D'$ است.

حال اگر درایه‌ی آخر $D'$ یعنی $D'_{d+1}$ را حذف کنیم، و مجددا درایه‌ $d+1$ از $B_i$ ها را حذف کرده تا دوباره به $A_i$ برسیم، به یک $x \in \mathbb{R}^d$ می‌رسیم که در $\bigcap_i^r A_i$ وجود دارد که همان حکم قسمت (الف) است.
\begin{latin}
	$\square$
\end{latin}

\end{document}t