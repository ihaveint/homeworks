\documentclass[a4paper,12pt]{article}
\usepackage{HomeWorkTemplate}
\usepackage{circuitikz}
\usepackage[shortlabels]{enumitem}
\usepackage{hyperref}
\usepackage{tikz}
\usepackage{listings}

\usetikzlibrary{shapes.multipart}

\usepackage{amsmath}
\usepackage{amssymb}
\usepackage{tcolorbox}
\usepackage{xepersian}
\settextfont{XB Niloofar}
\usetikzlibrary{arrows,automata}
\usetikzlibrary{circuits.logic.US}
\usepackage{changepage}
\newcounter{problemcounter}
\newcounter{subproblemcounter}
\setcounter{problemcounter}{1}
\setcounter{subproblemcounter}{1}
\newcommand{\problem}[1]
{
	\subsection*{
		پرسش
%		\arabic{problemcounter} 
%		\stepcounter{problemcounter}
%		\setcounter{subproblemcounter}{1}
		#1
	}
}
\newcommand{\subproblem}{
	\textbf{\harfi{subproblemcounter})}\stepcounter{subproblemcounter}
}


\begin{document}
\handout
{بهینه‌سازی ترکیبیاتی مقدماتی}
{مرتضی علیمی، هانی احمد زاده}
{نیم‌سال اول 1399\lr{-}1400}
{اطلاعیه}
{سروش زارع}
{97100405}
 {تمرین تحویلی 1}
\problem{1}
\subproblem

\paragraph{\color{blue}\lr{ proof for $\leftarrow$}}
اگر دو نقطه‌ی گوشه‌ای $A$ و $B$ مجاور باشند، طبق تعریف ابتدای سوال یعنی پاره خط واصل آن‌ها یک وجه یک بعدی از چند وجهی $P$ است. از طرفی طبق تعریف وجه می‌دانیم که هر وجه $P$ به صورت
\begin{align*}
F := P \cap \{x | c'^Tx = \delta \} 	
\end{align*}

قابل تعریف است که 
$c'^Tx \leq \delta$
یک نامساوی معتبر برای $P$ است.
از آنجایی که وجه 1 بعدی شامل دو نقطه‌ی گوشه‌ای است، بنابراین تنها نقاط گوشه‌ای آن همین $A$ و $B$ هستند و طبق تعریف وجه در بالا، بردار $c$ وجود دارد که $A$ و $B$ تنها نقاط گوشه‌ای بهینه‌ی مساله‌ی
\begin{align*}
min\{c^Tx | x \in P \}	
\end{align*}
می‌باشند. ($c$ را قرینه‌ی $c'$ موجود در نامساوی معتبر
$c'^Tx \leq \delta$
تعریف می‌کنیم و با این کار 
$c^Tx \geq -\delta$
خواهد بود و فقط در نقاط $A$ و $B$ مینیمم خواهد شد).

\begin{latin}
	$\square$
\end{latin}


\paragraph{\color{blue}\lr{ proof for $\rightarrow$}}

اگر بردار $c$ وجود داشته باشد به طوری که $A$ و $B$ تنها نقاط گوشه‌ای بهینه‌ی مساله‌ی
\begin{align*}
min\{c^Tx | x \in P \}	
\end{align*}
باشند و این مقدار مینیمم را با $\theta$ نشان دهیم، تمام نقاط چندوجهی $P$ در نامساوی
$c^Tx \geq \theta$
صدق می‌کنند.
بنابراین داریم:
\begin{align*}
c^Tx \geq \theta \rightarrow -c^Tx \leq 	-\theta
\end{align*}
بنابراین می‌توانیم وجه زیر را برای $P$ تعریف کنیم:
\begin{align*}
	F := min\{x | c'^Tx = \delta \} \ \ | \ \  c' = -c , \delta = -\theta
\end{align*}
پس نقاط گوشه‌ای $A$ و $B$ تنها نقاط گوشه‌ای برای وجه $F$ می‌باشند و در نتیجه $F$ یک وجه یک بعدی است که شامل نقاط گوشه‌ای $A$ و $B$ است و طبق تعریف ابتدای سوال، $A$ و $B$ دو نقطه‌ی گوشه‌ای مجاور در چندوجهی $P$ خواهند بود.
\begin{latin}
	$\square$
\end{latin}



\subproblem
اگر $M$ بردار مشخصه یک تطابق در گراف $G$ باشد، راس‌هایی از $G$ که $M$ آنها را درگیر می‌کند را را $f(M)$ نشان می‌دهیم. همچنین کل رئوس $G$ را با 
$V(G)$
 نشان می‌دهیم. طبق تعریف برای هر تطابق کامل مانند $M'$
  داریم:
\begin{align*}
f(M') = V(G)
\end{align*}
همچنین برای دو تطابق $M_1$ و $M_2$، 
$M_1 \cap M_2$
را اشتراک بردار مشخصه‌های این دو تطابق تعریف می‌کنیم.
\lemma{بردار مشخصه‌ی هر تطابق کامل مانند $M_1$، یک نقطه‌ی گوشه‌ای در $P$ است.}
\proof{}
\lemma{دو بردار مشخصه‌ی $M_1$ و $M_2$ دو نقطه‌ی گوشه‌ای مجاور هستند اگر و تنها اگر زیر گراف القایی $G$ با رئوس مربوط به $V(G) \backslash (M_1 \cap M_2)$ دقیقا ۲ تطابق کامل داشته باشد.}
\proof{}
در ابتدا قسمت (الف) این سوال را به عنوان فرض قبول می‌کنیم.
\paragraph{\color{blue}\lr{ proof for $\rightarrow$}}
اگر با ازای دو بردار مشخصه‌ی $M_1$ و $M_2$ شرط بالا برقرار باشد، می‌توانیم تعریف کنیم:
\begin{equation*}
  c_i := \left\{
  \begin{array}{@{}ll@{}}
    -2, & \text{if}\ (M_1 \cap M_2)_i = 1\\
    \infty , & \text{otherwise}
  \end{array}\right.
\end{equation*} 
توجه کنید که می‌توانیم به جای بی‌نهایت از هر عدد مثبت بزرگی استفاده کنیم.
حال ادعا می‌کنیم بردار $c$ ساخته شده در شرط قسمت (الف) صدق می‌کند.
با توجه به اینکه دنبال جواب‌های بهینه‌ی 
$min \{c^Tx | x \in P\}$
هستیم، و با توجه به اینکه کمترین ضرایب مربوط به $c$ در $M_1 \cap M_2$ تعریف شده است و بقیه‌ی ضرایب بسیار بزرگ هستند، تنها دو جواب $M_1$ و $M_2$ جواب‌های بهینه‌ی این مساله هستند و بنابراین طبق (الف) این دو نقطه، دو نقطه‌ی گوشه‌ای مجاور هستند.
\begin{latin}
	$\square$
\end{latin}

\paragraph{\color{blue}\lr{ proof for $\leftarrow$}}
عکس نقیض این حالت را اثبات می‌کنیم. یعنی اگر تعداد تطابق‌های کامل زیر گراف القایی $G$با رئوس متناظر با
$V(G) \backslash (M_1 \cap M_2)$
مخالف با ۲ باشد، اثبات می‌کنیم که $M_1$ و $M_2$ نقاط گوشه‌ای مجاور نیستند.
خود این حکم را نیز با برهان خلف اثبات می‌کنیم. یعنی فرض کنید زیر گراف کاهش یافته به رئوس
$V' := V(G) \backslash (M_1 \cap M_2)$
۲ تطابق کامل نداشته باشد ولی $M_1$ و $M_2$ دو تطابق کامل باشند. بنابراین می‌توانیم نتیجه بگیریم که
$V'$
اکیدا بیش از ۲ تطابق کامل دارد (وگرنه $M_1$ و $M_2$ تطابق کامل نبودند که تناقض است). این تطابق‌های کامل را 
$D_1, D_2, ... D_k$
که
$k > 2$
است نشان دهید.
بردار $c$ را مشابه با طرف دیگر اثبات تعریف کنید. 
حال تعریف می‌کنیم:
\begin{align*}
	C_i = D_i + (M_1 \cap M_2)
\end{align*}
در واقع هر کدام از $C_i$ ها بردار مشخصه‌ی تطابق کاملی است که یال‌های مربوط به $M_1 \cap M_2$ به همراه یال‌های مربوط به $D_i$ را دارد.
می‌توان دید که به ازای هر $x \in \{C_1,C_2,...C_k\}$  
مقدار
$c^Tx$
کمینه‌ی 
$min \{c^Tx | x \in P\}$
است و با توجه به اینکه $k > 2$، طبق (الف) نتیجه می‌گیریم که $M_1$ و $M_2$ دو نقطه‌ی گوشه‌ای مجاور نیستند. پس عکس نقیض را اثبات کردیم و حکم مد نظر اثبات شد، یعنی به ازای هر دو بردار مشخصه‌ی $M_1$ و $M_2$ که دو نقطه‌ی گوشه‌ای مجاور هستند، 
زیر گراف القایی $G$ با رئوس مربوط به $V(G) \backslash (M_1 \cap M_2)$ دقیقا ۲ تطابق کامل دارد.
\begin{latin}
	$\square$
\end{latin}

\end{document}